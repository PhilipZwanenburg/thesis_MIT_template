%%%%%%%%%%%%%%%%%%%%%%%%%%%%%%%%%%%%%%%%%%%%%%%%%%%%%%%%%%%%%%%%%%%%%%%%%%%%%%%%%%%%%%%%%%%%%%%%%%%%%%%%%%%%%%%%%%%%%%%%
\section{Abstract Functional Setting}

Much of the theory outlined below is borrowed directly from the works of Demkowicz et
al.~\cite{Demkowicz2014_overview,Demkowicz2017}. Of primary note, it is demonstrated that each DPG method can be
interpreted as the \emph{localization} of a method achieving optimal discrete stability through the choice of an
optimal conforming test space.

%%%%%%%%%%%%%%%%%%%%%%%%%%%%%%%%%%%%%%%%%%%%%%%%%%%%%%%%%%%%%%%%%%%%%%%%%%%%%%%%%%%%%%%%%%%%%%%%%%%%%%%%%%%%%%%%%%%%%%%%
\subsection{A Petrov-Galerkin Variational Methodology with Optimal Stability} \label{sec:PG_aspect_of_DPG}

Consider the \emph{linear} abstract variational problem
\begin{align} \label{eq:bilinear_abstract_infinite}
\text{Find}\ u \in \fspace{U}\ \text{such that}\
b(v,u) = l(v)\ \forall v \in \fspace{V},
\end{align}

where $\fspace{U}$ and $\fspace{V}$ denote the trial and test spaces, respectively, which are assumed to be Hilbert
spaces, and where the bilinear form $b(\cdot,\cdot)$ acting on $\fspace{V} \times \fspace{U}$ and the linear form
$l(\cdot)$ acting on $\fspace{V}$ correspond to a particular variational formulation. It is assumed that the bilinear form
satisfies a continuity condition with continuity constant $M$,
\begin{align}
| b(v,u) | \le M
||v||_{\fspace{V}}
||u||_{\fspace{U}},
\end{align}

and an inf-sup condition with inf-sup constant $\gamma$,
\begin{align} \label{eq:inf_sup_infinite}
\exists \gamma > 0 :
\adjustlimits\inf_{u\in\fspace{U}} \sup_{v\in\fspace{V}}
\frac{b(v,u)}{||v||_{\fspace{V}}||u||_{\fspace{U}}} \ge \gamma.
\end{align}

Further, it is assumed that the linear form is continuous and satisfies the following compatibility condition
\begin{align}
l(v) = 0\ \forall v \in \fspace{V}_0,\ \text{where}\ \fspace{V}_0 \coloneqq \{v \in \fspace{V} : b(v,u) = 0\ \forall u
\in \fspace{U} \}.
\end{align}

% Special characters in text: https://tex.stackexchange.com/a/8859
Then, by the Banach-Ne\v{c}as-Babu\v{s}ka theorem \makered{(Add reference to Brener-Scott/Ciarlet)},
~\eqref{eq:bilinear_abstract_infinite} has a unique solution, $u$, that depends continuously on the data,
\begin{align}
||u||_{\fspace{U}} \le \frac{M}{\gamma} ||l||_{\fspace{V}'},
\end{align}

where $\fspace{V}'$ denotes the dual space of $\fspace{V}$. Now let $\fspace{U}_h \subseteq \fspace{U}$ and $\fspace{V}_h
\subseteq \fspace{V}$ be finite dimensional trial and test spaces and consider the finite dimensional variation problem
\begin{align} \label{eq:bilinear_abstract_finite}
\text{Find}\ u_h \in \fspace{U}_h\ \text{such that}\
b(v_h,u_h) = l(v_h)\ \forall v_h \in \fspace{V}_h.
\end{align}

If the form satisfies the discrete inf-sup condition with inf-sup constant $\gamma_h$,
\begin{align} \label{eq:inf_sup_finite}
\exists \gamma_h > 0 :
\adjustlimits\inf_{u_h\in\fspace{U}_h} \sup_{v_h\in\fspace{V}_h}
\frac{b(v_h,u_h)}{||v_h||_{\fspace{V}_h}||u_h||_{\fspace{U}_h}} \ge \gamma_h,
\end{align}

then Babu\v{s}ka's theorem~\cite[Theorem \makeblue{2.2}]{Babuska1971} demonstrates that the discrete problem,
~\eqref{eq:bilinear_abstract_finite}, is well-posed with the Galerkin error satisfying the error estimate,
\begin{align} \label{eq:abstract_dpg_error}
||u-u_h||_{\fspace{U}} \le \frac{M}{\gamma_h} \inf_{w_h \in \fspace{U}_h} ||u-w_h||_{\fspace{U}}.
\end{align}

where the original constant in the bound, $\left( 1 + \frac{M}{\gamma_h} \right)$~\cite[eq. (\makeblue{2.14})]{Babuska1971}, has been
sharpened to $\frac{M}{\gamma_h}$ as demonstrated to be possible by Xu et al.~\cite[Theorem \makeblue{2}]{Xu2003}.
Generally, the well-posedness of the continuous problem does not imply the well-posedness of the discrete problem
(i.e.~\eqref{eq:inf_sup_infinite} $\centernot\implies$~\eqref{eq:inf_sup_finite}), and the fundamental motivation for
the DPG methodology is then to choose the test space such that the supremum in the discrete inf-sup
condition,~\eqref{eq:inf_sup_finite}, is obtained. \makered{(Potentially refer to where it is proven that DPG test
functions are chosen in this way (following the demonstration in Demkowicz et al.~\cite[Section 4.1]{Demkowicz2017}))}

A case of particular interest is then when the continuity and discrete inf-sup constants can be made equal,
\begin{align} \label{eq:equal_m_gamma}
M = \gamma_h,
\end{align}

so that the error incurred by the discrete approximation in~\eqref{eq:abstract_dpg_error} is smallest. As it is not
immediately clear which norms should be selected for the trial and test spaces, the simplest strategy is to let the norm
be chosen as that which is naturally induced by the problem such that~\eqref{eq:equal_m_gamma} is satisfied. Bui-Thanh
et al.~\cite[Theorem \makeblue{2.6}]{BuiThanh2013} have proven that $M = \gamma = 1$ if
\begin{align} \label{eq:bui_thanh_2-6}
\exists v_u \in \fspace{V}\ \setminus \{0\} :
b(v_u,u) = ||v_u||_{\fspace{V}} ||u||_{\fspace{U}}\ \forall u \in \fspace{U} \setminus \{0\},
\end{align}

where $v_u$ is termed an optimal test function for the trial function $u$. Further, assuming
that~\eqref{eq:bui_thanh_2-6} holds,~\eqref{eq:equal_m_gamma} is satisfied when the discrete test space is defined by
\begin{align} \label{eq:def_discrete_test_space}
\fspace{V} \supset \fspace{V}_h 
= \spantext{ \{ v_{u_h} \in \fspace{V} : u_h \in \fspace{U}_h \subseteq \fspace{U},\ b(v_{u_h},u_h) }
= ||v_{u_h}||_{\fspace{V}} ||u_h||_{\fspace{U}} \};
\end{align}

see Bui-Thanh et al.~\cite[Lemma \makeblue{2.8}]{BuiThanh2013}. Defining the map from trial to test space, $T :
\fspace{U} \ni u \rightarrow Tu \coloneqq v_{Tu} \in \fspace{V} $, by
\begin{align}
(v,Tu)_{\fspace{V}} = b(v,u),
\end{align}

where $(\cdot,\cdot)_{\fspace{V}}$ represents the test space inner product, then the discrete test
space,~\eqref{eq:def_discrete_test_space}, is equivalently defined as
\begin{align} \label{eq:def_discrete_test_space_alt}
\fspace{V}_h = \spantext{ \{ v_{T u_h} \in \fspace{V} : u_h \in \fspace{U}_h \} };
\end{align}

see Bui-Thanh et al.~\cite[Theorem \makeblue{2.9}]{BuiThanh2013}. Defining the Riesz operator for the test inner
product,
\begin{align}
R_{\fspace{V}} : \fspace{V} \ni v \rightarrow (v,\cdot) \in \fspace{V}',
\end{align}

which is an isometric isomorphism (\cite[Theorem \makeblue{4.9-4}]{Ciarlet2013}), the test functions
spanning $\fspace{V}_h$, which are henceforth referred to as \emph{optimal} test
functions, can be computed through the inversion of the Riesz operator by solving the auxiliary variational problem
\begin{align} \label{eq:auxiliary_opt_v}
\text{Find}\ v_{T u_h} \in \fspace{V}\ \text{such that}\
(w,v_{T u_h})_{\fspace{V}} = b(w,u_h),\ \forall w \in \fspace{V}.
\end{align}

%%%%%%%%%%%%%%%%%%%%%%%%%%%%%%%%%%%%%%%%%%%%%%%%%%%%%%%%%%%%%%%%%%%%%%%%%%%%%%%%%%%%%%%%%%%%%%%%%%%%%%%%%%%%%%%%%%%%%%%%
\subsection{DPG as a Localization of the Optimal Conforming PG Methodology} \label{sec:dpg_abstract_setting}

Note that no assumptions regarding the conformity of the trial and test spaces were imposed
in~\autoref{sec:PG_aspect_of_DPG}. Specifically, the specification of the `D' (discontinuous) in DPG, referring to a
discontinuous test space, has not yet been made and the methodology described is thus of a general Petrov-Galerkin
form. Denote the trial \emph{graph} space over the domain $\Omega$, $\fspace{H}_b(\Omega)$, as that of the solution
of~\eqref{eq:bilinear_abstract_infinite},
\begin{align}
\fspace{H}_b(\Omega) \coloneqq
\{ u \in (L^2(\Omega))^n : b(v,u) \in (L^2(\Omega))^n\ \forall v \in \fspace{V} \},
\end{align}

where $n$ denotes the number of scalar variables. Integration by parts of~\eqref{eq:bilinear_abstract_infinite} leads to
the formal $L^2$-adjoint and a bilinear form representing the boundary terms
\begin{align}
b(v,u) = b^*(v,u) + c(\text{tr}_A^* v, \text{tr}_A u)
\end{align}

where $v$ is in the graph space for the adjoint
\begin{align}
\fspace{H}_b^*(\Omega) \coloneqq
\{ v \in (L^2(\Omega))^n : b^*(v,u) \in (L^2(\Omega))^n\ \forall u \in \fspace{U} \}.
\end{align}

See Demkowicz et al.~\cite[eq. (\makeblue{4.18})]{Demkowicz2014_overview} for a concrete example of these operators.
When setting $\fspace{V} = \fspace{H}_b^*(\Omega)$, we say that the test space is $\fspace{H}_b$-conforming.
\\~

As the eventual goal of the methodology is to solve~\eqref{eq:bilinear_abstract_finite} over a tessellation,
$\mathcal{T}_h$, of the discretized domain, $\Omega_h$, consisting of elements (referred to as volumes) $\fe{V}$, we
note that using an $\fspace{H}_b$-conforming test space results in each of the optimal test functions computed
by~\eqref{eq:auxiliary_opt_v} having global support (i.e. potentially over all of $\Omega_h$).  To make the methodology
practical, broken energy spaces are introduced such that the required inversion of the Riesz operator can be done
elementwise, \emph{localizing}~\eqref{eq:auxiliary_opt_v},
\begin{align} \label{eq:auxiliary_opt_v_infinite}
\text{Find}\ v_{T u_h} \in \fspace{V}(\fe{V})\ \text{such that}\
(w,v_{T u_h})_{\fspace{V}(\fe{V})} = b(w,u_h),\ \forall w \in \fspace{V}(\fe{V})
\end{align}

where 
\begin{align}
\fspace{V}(\Omega_h) 
&\coloneqq
\{ v \in L^2(\Omega) : v|_{\fe{V}} \in \fspace{H}_b^*(\fe{V})\ \forall \fe{V} \in \mathcal{T}_h \}, \\
(w,v)_{\fspace{V}(\Omega_h)}
&\coloneqq
\sum_{\fe{V}} (w|_{\fe{V}},v|_{\fe{V}})_{\fspace{V}(\fe{V})}
\end{align}

and $\fspace{V}(\fe{V})$ is the volume test space. Finally, it must be noted that the variational problem for the test
functions,~\eqref{eq:auxiliary_opt_v_infinite}, is infinite dimensional. In practice, it must be solved approximately, for
\textit{approximate optimal test functions}, in an approximate volume test space $\afspace{V} \subseteq \fspace{V}$,
\begin{align} \label{eq:auxiliary_opt_v_finite}
\text{Find}\ \tilde{v}_{T u_h} \in \afspace{V}(\fe{V})\ \text{such that}\
(\tilde{w},\tilde{v}_{T u_h})_{\afspace{V}(\fe{V})} = b(\tilde{w},u_h),\ \forall \tilde{w} \in \afspace{V}(\fe{V}),
\end{align}

where the corresponding approximate optimal test space is, analogous to~\eqref{eq:def_discrete_test_space_alt}, defined by
\begin{align} \label{eq:def_discrete_test_space_alt_approx}
\afspace{V}_h = \spantext{ \{ \tilde{v}_{T u_h} \in \afspace{V} : u_h \in \fspace{U}_h \} }.
\end{align}

\makered{Potentially add comment regarding account for the approximation of optimal test functions \cite[eqs. (4.30) -
(4.32)]{Demkowicz2014_overview}.}

Noting that the test graph space is a subset of $(L^2(\Omega))^n$, it has been shown that the PG methodology
of~\autoref{sec:PG_aspect_of_DPG} is, in fact, a subset of this practical DPG methodology where $L^2$-conforming test
spaces are used as shown in the proof of~\autoref{prop:pg_subset_of_dpg}. This however comes at the cost of introducing
trace unknowns over the interior volume boundaries, in addition to the already existing trace unknowns on the domain
boundary, both of which are subsequently denoted by $\hat{u}$.

Defining the group variable $\varg{u} \coloneqq (u,\hat{u})$ containing both the solution components in $L^2$ as well as
those defined on traces (boundary and internal),
we separate the bilinear form into the following components
\begin{align} \label{eq:bilinear_decomposition}
b(v,\varg{u})
\coloneqq b(v,(u,\hat{u})) \coloneqq \bar{b}(v,\varg{w}) + \left<\left< v,\hat{w} \right>\right>
\end{align}

where $\bar{b}(v,\varg{w})$ includes all terms present in the PG methodology outlined in~\autoref{sec:PG_aspect_of_DPG} and
$\left<\left< v,\hat{w} \right>\right>$ accounts for newly introduced terms arising as a result of the usage of the
broken tests space. Above, $w$ includes the solution component in $L^2$, $u$, as well as the trace component on the
domain boundary while $\hat{w}$ includes only the trace component on the internal volume boundaries. Following the
previous notation, discrete solution variables are represented by $\varg{w}_h \in \fspace{U}_h \times \ftspace{U}_h$ and
$\hat{w}_h \in \ftspace{W}_h \subset \ftspace{U}_h$. Defining the weakly
conforming optimal test space as
\begin{align} \label{eq:weakly_conforming_approx_test_space}
\bar{V}_h = \{ v \in \afspace{V}_h : \left<\left< v,\hat{w}_h \right>\right> = 0\ \forall \hat{w}_h \in \aftspace{W} \},
\end{align}

we have the following

\begin{proposition}[PG Test Space as a Strict Subset of DPG Test Space] \label{prop:pg_subset_of_dpg}
$\bar{V}_h \subset \afspace{V}_h.$
\end{proposition}

\begin{proof}

We briefly reproduce the proof of Demkowicz et al.~\cite[Section \makeblue{6}]{Demkowicz2014_overview}. As
$\afspace{V}$ is a finite dimensional Hilbert space and $\afspace{V}_h \subseteq \afspace{V}$, the direct sum
theorem~\cite[Theorem \makeblue{4.5-2}]{Ciarlet2013} allows for its decomposition as
\begin{align}
\afspace{V} = \afspace{V}_h + \afspace{V}_h^{\bot}
\end{align}

where $\afspace{V}_h^{\bot}$ is the $\afspace{V}$-orthogonal complement of $\afspace{V}_h$ in $\afspace{V}$. This can be
seen from 
\begin{align}
\afspace{V}_h^{\bot}
\coloneqq\ & \{ \tilde{v} \in \afspace{V} :
(\tilde{x},\tilde{v})_{\afspace{V}(\fe{V})} = b(\tilde{x},\varg{u}_h),\ \forall \varg{u}_h \in \fspace{U} \setminus
\fspace{U}_h,\ \forall \tilde{x} \in \afspace{V}(\fe{V}) \}
 && (\text{using}~\eqref{eq:auxiliary_opt_v_finite}) \nonumber \\
=\ & \{ \tilde{v} \in \afspace{V} :
(\tilde{x},\tilde{v})_{\afspace{V}(\fe{V})} = 0,\ \forall \tilde{x} \in \afspace{V}(\fe{V}) \}.
 && (\text{by Galerkin orthogonality}) \nonumber
\end{align}

\makered{Think about the implication above that $(\tilde{x},\tilde{v})_{\afspace{V}(\fe{V})} \ne 0\ \forall \tilde{v}
\in \afspace{V}_h, \ \forall \tilde{x} \in \afspace{V}(\fe{V})$.}

Let $\bar{V}_h \ni \bar{v} = \{ v \in \afspace{V} : (\tilde{x},v)_{\afspace{V}(\fe{V})} = \bar{b}(\tilde{x},\varg{w}_h)\ \forall
\tilde{x} \in \afspace{V}(\fe{V}) \}$. Since $\bar{v} \in \afspace{V}$, it can be decomposed as
\begin{align}
\bar{v} = \bar{v}_h + \bar{v}_h^{\bot},\ \bar{v}_h \in \afspace{V}_h,\ \bar{v}_h^{\bot} \in \afspace{V}_h^{\bot}.
\end{align}

Since, $T\hat{w}_h \in \afspace{V}_h$, it follows that
\begin{align}
0 
=\ & (\bar{v}_h^{\bot},T\hat{w}_h)_{\afspace{V}(\fe{V})} && (\text{using}\ \afspace{V}\text{-orthogonality}) \nonumber \\
=\ & b(\bar{v}_h^{\bot},(0,\hat{w}_h)) && (\text{using}~\eqref{eq:auxiliary_opt_v_finite}) \nonumber \\
=\ & \left<\left< \bar{v}_h^{\bot},\hat{w}_h \right>\right> && (\text{using}~\eqref{eq:bilinear_decomposition}) \label{eq:weak_continuous_v}
\end{align}

and consequently that $\bar{v}_h^{\bot} \in \bar{V}_h$. Because $T\varg{w}_h \in \afspace{V}_h$, then, as above,
\begin{align}
0 = \bar{b}(\bar{v}_h^{\bot},\varg{w}_h).
\end{align}

By definition of $\bar{v}$
\begin{align}
0
=\ &
(\bar{v}_h^{\bot},\bar{v})_{\afspace{V}(\fe{V})} && (\text{by definition}) \nonumber \\
=\ &
(\bar{v}_h^{\bot},\bar{v}_h^{\bot})_{\afspace{V}(\fe{V})} && (\text{using}\ \afspace{V}\text{-orthogonality}). \nonumber
\end{align}

and it is immediate that $\bar{v} = \bar{v}_h \in \afspace{V}_h$.

\end{proof}

If the optimal conforming PG methodology,~\autoref{sec:PG_aspect_of_DPG}, and the DPG methodology are both uniquely
solvable, then their solutions are the same because substitution of conforming test functions into the DPG formulation
immediately recovers the PG formulation.
