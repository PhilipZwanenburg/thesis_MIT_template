%%%%%%%%%%%%%%%%%%%%%%%%%%%%%%%%%%%%%%%%%%%%%%%%%%%%%%%%%%%%%%%%%%%%%%%%%%%%%%%%%%%%%%%%%%%%%%%%%%%%%%%%%%%%%%%%%%%%%%%%
\section{A Concrete Example: Linear Advection}

Consider the steady linear advection equation as a model problem
\begin{subequations}
\begin{align}
& \varv{b} \cdot \nabla \vars{u} = s && \text{in}\ \Omega, \label{eq:linear_advection_omega} \\
& u = u_{\Gamma}\ && \text{on}\ \Gamma \coloneqq \{ \varv{x} \in \partial\Omega : \varv{\hat{n}} \cdot b < 0 \},
\end{align}
\end{subequations}

where $\varv{b}$ is the advection velocity and $\varv{\hat{n}}$ is the outward pointing normal vector. Partitioning the
domain with $N_v$ non-overlapping volumes, $\fe{V}$, with faces, $\fe{F}_v \coloneqq \partial \fe{V}_v$, such that
$\Omega_h = \sum_{v=1}^{N_v}$, \makered{(Check notation with Brenner)},~\eqref{eq:linear_advection_omega} is multiplied
by a test function $v$ and integrated by parts to give the bilinear and linear forms
\begin{subequations} \label{eq:variational_form_advection}
\begin{align}
b(v,\varg{u}) 
=\ & \sum_{v=1}^{N_v} \int_{\fe{V}_v} - \nabla v \cdot \varv{b} u\ d\fe{V_v} + \int_{\fe{F}_v \cap \Gamma^i} v
\varv{\hat{n}} \cdot \varv{b} u^*\ d\fe{F}_v, \label{eq:bilinear_pg_advection}\\
l(v)
=\ & \sum_{v=1}^{N_v} \int_{\fe{V}_v} v s\ d\fe{V_v} + \int_{\fe{F}_v \cap \Gamma} v \varv{\hat{n}} \cdot \varv{b}
u_{\Gamma}\ d\fe{F}_v,
\end{align}
\end{subequations}

where $\Gamma^i$ is the set of all internal volume faces (i.e. all faces excluding those in $\Gamma$) and where the
single-valued solution trace unknowns, $u^*$, have been introduced as part of the group variable $\varg{u} \coloneqq
(u,u^*)$.~\eqref{eq:bilinear_pg_advection} can be expressed more compactly as
\begin{align} \label{eq:bilinear_pg_advection_jump}
b(v,\varg{u}) 
=\ & \sum_{v=1}^{N_v} \int_{\fe{V}_v} - \nabla v \cdot \varv{b} u\ d\fe{V_v}
+ \frac{1}{2} \int_{\fe{F}_v} \jump{v} \varv{\hat{n}} \cdot \varv{b} u^*\ d\fe{F}_v,
\end{align}

after introducing the \textit{jump} operator, $\jump{v} = v^- - v^+ $, with ``$-$'' and ``$+$'' referring to the
internal and external volumes adjacent to the face and with the additional specification of $v^+ = \pm v^-$ on
inflow/outflow boundaries, respectively. Following the motivation of pursuing norms where the continuity and inf-sup
constants are equal, the Cauchy-Schwarz inequality can be applied to~\eqref{eq:bilinear_pg_advection_jump} to obtain
\begin{align} \label{eq:cauchy_scwarz_advection}
b(v,\varg{u})
\le\ & \sum_{v=1}^{N_v}
|| - \nabla v \cdot \varv{b} ||_{L^2(\fe{V}_v)} || u ||_{L^2(\fe{V}_v)}
+
\frac{1}{2} ||\jump{v}||_{L^2_{\varv{\hat{n}}\cdot\varv{b}}(\fe{F}_v)}
||u^*||_{L^2_{\varv{\hat{n}}\cdot\varv{b}}(\fe{F}_v)} \\
\le\ &
\underbrace{\left(
\sum_{v=1}^{N_v} || - \nabla v \cdot \varv{b} ||^2_{L^2(\fe{V}_v)} + \frac{1}{\sqrt{2}} ||\jump{v}||^2_{L^2_{\varv{\hat{n}}\cdot\varv{b}}(\fe{F}_v)}
\right)^\frac{1}{2}}_{||v||_\fspace{V}}
\times
\underbrace{\left(
\sum_{v=1}^{N_v} || u ||^2_{L^2(\fe{V}_v)} + \frac{1}{\sqrt{2}} ||u^*||^2_{L^2_{\varv{\hat{n}}\cdot\varv{b}}(\fe{F}_v)}
\right)^\frac{1}{2}}_{||u||_\fspace{U}}.
\end{align}
% Note that the above is a norm on $\fspace{V}$ because constants are accounted for by the trace term.

Above, $\varg{u} \in \fspace{U} = L^2(\Omega_h) \times L^2_{\varv{\hat{n}}\cdot\varv{b}}(\Gamma_h^i \cup
\Gamma_h)$ and $v \in \fspace{V} = H^1_\vect{b}(\Omega_h)$ where the spaces are defined according to
\begin{align}
L^2(\Omega_h)
=\ & \{ u : u \in L^2(\fe{V}_v)\ \forall \fe{V}_v \in \Omega_h \}, \\
L^2_{\varv{\hat{n}}\cdot\varv{b}}(\Gamma_h^i \cup \Gamma_h)
=\ & \{ \hat{u} : \hat{u} \in L^2_{\varv{\hat{n}}\cdot\varv{b}}(\fe{F}_v)\ \forall \fe{V}_v \in \Omega_h \}, \\
H^1_{\vect{b}}(\Omega_h)
=\ & \{ v : v \in L^2(\Omega_h),\ \vect{b} \cdot \nabla v \in L^2(\Omega_h) \},
\end{align}

where
\begin{align}
L^2_{\varv{\hat{n}}\cdot\varv{b}}(\fe{F}_v)
=\ & \left\{ \hat{w} : || \hat{w} ||^2_{L^2_{\varv{\hat{n}}\cdot\varv{b}}(\fe{F}_v)} = \int_{\fe{F}_v} |\vect{\hat{n}} \cdot
\vect{b}| |\hat{w}|^2 d\fe{F}_v < \infty \right\}.
\end{align}

\makered{Note that the above is not a norm because of the possibility of $\varv{\hat{n}}\cdot\varv{b} = 0$. Is this a
problem?} The equality in~\eqref{eq:cauchy_scwarz_advection} is attained, as required by~\eqref{eq:bui_thanh_2-6}, if
the test functions are chosen such that
\begin{align}
u =\ & -\nabla v_u \cdot \vect{b} && \text{in}\ \fe{V}_v, \\
u^* =\ & \sign{\vect{\hat{n}} \cdot \vect{b}} \jump{v_u} && \text{on}\ \fe{F}_v.
\end{align}

In general, for basis functions of the form $\varg{\hat{\phi}} = (0,\hat{\phi}) \in \fspace{U}$,
\begin{align}
0 =\ & -\nabla v_{\hat{\phi}} \cdot \vect{b} && \text{in}\ \fe{V}_v, \label{eq:optimal_test_adv_trace_0} \\
\hat{\phi} =\ & \sign{\vect{\hat{n}} \cdot \vect{b}} \jump{v_{\hat{\phi}}} && \text{on}\ \fe{F}_v,
\label{eq:optimal_test_adv_trace_1}
\end{align}

and for basis functions $\varg{\phi} = (\phi,0) \in \fspace{U}$,
\begin{align}
\phi =\ & -\nabla v_{\phi} \cdot \vect{b} && \text{in}\ \fe{V}_v, \label{eq:optimal_test_adv_sol} \\
0 =\ & \sign{\vect{\hat{n}} \cdot \vect{b}} \jump{v_{\phi}} && \text{on}\ \fe{F}_v.
\label{eq:optimal_test_adv_sol_conforming}
\end{align}

Considering the test function for the $L^2$ component of the solution, it can be noted
that~\eqref{eq:optimal_test_adv_sol_conforming} imposes a conformity constraint on the test space, resulting in a
specific PG method from~\autoref{sec:PG_aspect_of_DPG}. However, recalling~\autoref{prop:pg_subset_of_dpg}, the same
solution is obtained using the practical DPG methodology of~\autoref{sec:dpg_abstract_setting}, and this can be achieved
by omitting the conformity constraint and limiting the support of each of the optimal test functions,
\begin{align}
\support{v_{\hat{\phi} \in \fe{F}_v}} =\ & \{ \fe{V}_i \in \Omega_h : \fe{V}_i \cap \fe{F}_v \ne \emptyset \}, \\
\support{v_{\phi \in \fe{V}_v}} =\ & \{ \fe{V}_i \in \Omega_h : \fe{V}_i \cap \fe{V}_v \ne \emptyset \}.
\end{align}

\begin{proposition}
If the localized test norm for the optimal test functions is given by 
\begin{align}
(w,v)_{\fspace{V}(\fe{V})}
= 
\int_{\fe{V}_v} (\nabla w \cdot \vect{b})(\nabla v \cdot \vect{b}) d\fe{V}_v
+
\int_{\fe{F}_v} \jump{w} |\vect{\hat{n}} \cdot \vect{b}| \jump{v} d\fe{F}_v
\end{align}
\makered{Missing one additional constraint above. Think of elegant way to add constraint to average value of downwind.}

for all volumes and faces in the support of the local test function, then the computed test functions for the trace
unknowns satisfy~\eqref{eq:optimal_test_adv_trace_0} and~\eqref{eq:optimal_test_adv_trace_1} exactly (for polynomial
test functions) and the trace solution computed after substituting the test functions into the variational
form,~\eqref{eq:variational_form_advection}, is also exact \makered{(in 1D)}.

\end{proposition}

\begin{proof}
So far only showed this numerically for a p1 1D modal basis.
\end{proof}


\makered{Where are we going with this?}
\begin{itemize}
\item Bui-Thanh2013's continuous method results in the recovery of the solution as the propagation of the boundary along
the characteristics.
\item Using the localized test norm: H1-semi + trace term, the trace test functions computed are exactly those
satisfying the constraints above. Inserting these into the bilinear form results in pointwise exact solution for the
trace unknowns (1D).
\item Equivalent with DG using lowest order test?
\end{itemize}
\makered{Ask Legrand if he is interested in this result once finished typesetting (and likely after 2D attempt is made).}


