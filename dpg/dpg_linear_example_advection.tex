%%%%%%%%%%%%%%%%%%%%%%%%%%%%%%%%%%%%%%%%%%%%%%%%%%%%%%%%%%%%%%%%%%%%%%%%%%%%%%%%%%%%%%%%%%%%%%%%%%%%%%%%%%%%%%%%%%%%%%%%
\section{A Concrete Example: Linear Advection}

Consider the steady linear advection equation as a model problem
\begin{subequations}
\begin{align}
& \varv{b} \cdot \nabla \vars{u} = s && \text{in}\ \Omega, \label{eq:linear_advection_omega} \\
& u = u_{\Gamma^{\text{i}}}\ && \text{on}\ \Gamma^{\text{i}} \coloneqq \{ \varv{x} \in \partial\Omega : \varv{\hat{n}} \cdot \vect{b} < 0 \},
\end{align}
\end{subequations}

where $\varv{b}$ is the advection velocity and $\varv{\hat{n}}$ is the outward pointing normal vector. Partitioning the
domain into non-overlapping volumes, $\fe{V}$, with faces, $\fe{F} \coloneqq \partial
\fe{V}$,~\eqref{eq:linear_advection_omega} is multiplied by a test function $v$ and integrated by parts to give the
bilinear and linear forms
\begin{subequations} \label{eq:variational_form_advection}
\begin{align}
b(v,\varg{u}) 
=\ & \sum_{\fe{V}}
\int_{\fe{V}} - \nabla v \cdot \varv{b} u\ d\fe{V} +
\int_{\fe{F} \setminus \Gamma^{\text{i}}} v f^*\ d\fe{F}, \label{eq:bilinear_pg_advection} \\
l(v)
=\ & \sum_{\fe{V}} \int_{\fe{V}} v s\ d\fe{V} + \int_{\fe{F} \cap \Gamma^{\text{i}}} v f_{\Gamma^{\text{i}}}\ d\fe{F},
\end{align}
\end{subequations}

where $f_{\Gamma^{\text{i}}} = \varv{\hat{n}} \cdot \varv{b} u_{\Gamma^{\text{i}}}$ and where the single-valued trace
normal fluxes, $f^* \coloneqq \varv{\hat{n}} \cdot \varv{b} u|_{\fe{F}}$, have been introduced as part of the group
variable $\varg{u} \coloneqq (u,f^*)$. The selection of $f^*$ instead of $u^*$ as the trace unknown is made because of
the possible degeneration of $\varv{\hat{n}} \cdot \varv{b}$. ~\eqref{eq:bilinear_pg_advection} can be expressed more
compactly as
\begin{align} \label{eq:bilinear_pg_advection_jump}
b(v,\varg{u}) 
=\ & \sum_{\fe{V}} \int_{\fe{V}} - \nabla v \cdot \varv{b} u\ d\fe{V}
+ \frac{1}{2} \int_{\fe{F}} \jump{v} f^*\ d\fe{F},
\end{align}

after introducing the \textit{jump} operator, $\jump{v} = v^- - v^+ $, with ``$-$'' and ``$+$'' referring to the
volumes adjacent to the face with the normal vector pointing outwards/inwards, respectively, and with the additional
specification of $v^+ = \pm v^-$ on inflow/outflow boundaries, respectively. Following the motivation of pursuing norms
where the continuity and inf-sup constants are equal, the Cauchy-Schwarz inequality can be applied
to~\eqref{eq:bilinear_pg_advection_jump} to obtain
% Cauchy-Swarz: https://en.wikipedia.org/wiki/Cauchy%E2%80%93Schwarz_inequality
\begin{align} \label{eq:cauchy_scwarz_advection}
b(v,\varg{u})
\le\ & \sum_{\fe{V}}
|| - \nabla v \cdot \varv{b} ||_{L^2(\fe{V})} || u ||_{L^2(\fe{V})}
+
\frac{1}{2} ||\jump{v}||_{L^2(\fe{F})} ||f^*||_{L^2(\fe{F})} \\
\le\ &
\underbrace{\left(
\sum_{\fe{V}} || - \nabla v \cdot \varv{b} ||^2_{L^2(\fe{V})} + \frac{1}{\sqrt{2}} ||\jump{v}||^2_{L^2(\fe{F})}
\right)^\frac{1}{2}}_{||v||_\fspace{V}}
\times
\underbrace{\left(
\sum_{\fe{V}} || u ||^2_{L^2(\fe{V})} + \frac{1}{\sqrt{2}} ||f^*||^2_{L^2(\fe{F})}
\right)^\frac{1}{2}}_{||u||_\fspace{U}}.
\end{align}
% Note that the above is a norm on $\fspace{V}$ because constants are accounted for by the trace term.

Above, $\varg{u} \in \fspace{U} = L^2(\Omega_h) \times L^2(\Gamma^+_h)$, and $v \in \fspace{V} = H^1_\vect{b}(\Omega_h)$
where the $h$ subscripts denote discretization, $\Gamma^+_h$ is the union of all volumes faces
including those of the boundary and the internal skeleton, and where the spaces are defined according to
\begin{align}
L^2(\Omega_h)
=\ & \{ u : u \in L^2(\fe{V})\ \forall \fe{V} \in \Omega_h \}, \\
L^2(\Gamma^+_h)
=\ & \{ \hat{u} : \hat{u} \in L^2(\fe{F})\ \forall \fe{V} \in \Omega_h \}, \\
H^1_{\vect{b}}(\Omega_h)
=\ & \{ v : v \in L^2(\Omega_h),\ \vect{b} \cdot \nabla v \in L^2(\Omega_h) \},
\end{align}

Note that the space chosen above for $\fspace{U}$ implies a weak imposition of the inflow boundary condition, and it is
imagined that a \textit{ghost} volume neighbouring the inflow boundaries allows for the imposition of the boundary
conditions. Equality in~\eqref{eq:cauchy_scwarz_advection} is attained, as desired due to~\eqref{eq:bui_thanh_2-6}, if
the test functions are chosen such that
\begin{align}
u =\ & -\nabla v_u \cdot \vect{b} && \text{in}\ \fe{V}, \\
f^* =\ & \jump{v_u} && \text{on}\ \fe{F}.
\end{align}

In general, for basis functions of the form $\varg{\hat{\phi}} = (0,\hat{\phi}) \in \fspace{U}$,
\begin{align}
0 =\ & -\nabla v_{\hat{\phi}} \cdot \vect{b} && \text{in}\ \fe{V}, \label{eq:optimal_test_adv_trace_0} \\
\hat{\phi} =\ & \jump{v_{\hat{\phi}}} && \text{on}\ \fe{F},
\label{eq:optimal_test_adv_trace_1}
\end{align}

and for basis functions $\varg{\phi} = (\phi,0) \in \fspace{U}$,
\begin{align}
\phi =\ & -\nabla v_{\phi} \cdot \vect{b} && \text{in}\ \fe{V}, \label{eq:optimal_test_adv_sol} \\
0 =\ & \jump{v_{\phi}} && \text{on}\ \fe{F}.
\label{eq:optimal_test_adv_sol_conforming}
\end{align}

Considering the test function for the $L^2$ component of the solution, it can be noted
that~\eqref{eq:optimal_test_adv_sol_conforming} imposes a conformity constraint on the test space, resulting in a
specific PG method from~\autoref{sec:PG_aspect_of_DPG}. However, recalling~\autoref{prop:pg_subset_of_dpg}, the same
solution is obtained using the practical DPG methodology of~\autoref{sec:dpg_abstract_setting}, and this can be achieved
by omitting the conformity constraint and limiting the support of each of the optimal test functions,
\begin{align}
\support{v_{\hat{\phi} \in \fe{F}}}
=\ & \{ \fe{V}_i \in \Omega_h : \fe{V}_i \cap \fe{F} \ne \emptyset \}
\coloneqq \{ \fe{V}_{\fe{F}}^- \cup \fe{V}_{\fe{F}}^+ \}, \\
\support{v_{\phi \in \fe{V}}} =\ & \{ \fe{V}_i \in \Omega_h : \fe{V}_i \cap \fe{V} \ne \emptyset \},
\end{align}

such that they can be computed in a local manner.

\begin{proposition} \label{prop:test_norm_for_characteristic_test}
Defining the ``-'' volume in relation to a specific face as that which satisfies $\vect{\hat{n}} \cdot \vect{b} \ge 0$,
then given the localized test norm (inner product)
\begin{align}
(w,v)_{\fspace{V}(\fe{V})}
= 
\int_{\fe{V}} (\nabla w \cdot \vect{b})(\vect{b} \cdot \nabla v) d\fe{V}
+
\int_{\fe{F}} \jump{w} |\vect{\hat{n}} \cdot \vect{b}| \jump{v} d\fe{F}
+
\int_{\fe{F} \cap \fe{V}_{\fe{F}}^+} w |\vect{\hat{n}} \cdot \vect{b}| v d\fe{F}
\end{align}
\makered{Note that the above is not a norm when $\vect{\hat{n}} \cdot \vect{b} = 0$. This is not a problem for the
method as the trace solution component is not required on such faces, but is a problem for the rigour. Change the norm
on such faces? Add an L2 component?}

where integration is performed over all volumes and faces in the support of the local test function, then the computed test functions for the trace unknowns satisfy
\begin{align}
||v_{\hat{\phi}}-v_{\hat{\phi}_h}||_{{L^2(\Omega_h)}}
\le
\inf_{w_{\hat{\phi}_h} \in L^2(\fe{V}_{\fe{F}}^- \cup \fe{V}_{\fe{F}}^+)}
||v_{\hat{\phi}}-w_{\hat{\phi}_h}||_{{L^2(\Omega_h)}},
\end{align}

where the exact test functions satisfy~\eqref{eq:optimal_test_adv_trace_0} and~\eqref{eq:optimal_test_adv_trace_1}.
Further, the computed trace unknowns obtained by solving~\eqref{eq:bilinear_abstract_infinite} after substituting
$v_{\hat{\phi}}$ into~\eqref{eq:variational_form_advection} satisfy
\begin{align}
||u^*-\hat{u}_h||_{{L^2(\Gamma^+_h)}}
\le
\inf_{\hat{w}_h \in {L^2(\Gamma^+_h)}}
||u^*-\hat{w}_h||_{{L^2(\Gamma^+_h)}}.
\end{align}

\end{proposition}

\begin{proof}
Solving for the approximate optimal test functions using~\eqref{eq:auxiliary_opt_v_finite} with $\varg{u}_h =
(0,\hat{\phi})$,
\begin{align}
1
\end{align}

\end{proof}

\begin{remark}
The physical intuition behind the choice of test inner product in~\autoref{prop:test_norm_for_characteristic_test} is
that the resultant test functions serve to propagate solutions taking the form $\hat{\phi}$ from inflow to outflow
volume faces along the flow characteristics (in the advection direction) while simultaneously adding the $L^2$
projection of the source onto the trace basis function lifted into the volume along the characteristics. \makered{(Add
figure).}
\end{remark}


\makered{Where are we going with this?}
\begin{itemize}
\item Bui-Thanh2013's continuous method results in the recovery of the solution as the propagation of the boundary along
the characteristics.
\item Using the localized test norm: H1-semi + trace term, the trace test functions computed are exactly those
satisfying the constraints above. Inserting these into the bilinear form results in pointwise exact solution for the
trace unknowns (1D).
\item Equivalent with DG using lowest order test?
\end{itemize}
\makered{Ask Legrand if he is interested in this result once finished typesetting (and likely after 2D attempt is made).}


