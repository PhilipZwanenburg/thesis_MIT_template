\documentclass{article}

% To compile the pdf, execute the following in the terminal:
% $ pdflatex ${file_name}.tex
% $ bibtex ${file_name}
% $ sage ${file_name}.sagetex.sage
% $ pdflatex ${file_name}.tex

\usepackage{sagetex}
\setlength{\sagetexindent}{10ex}

\usepackage{hyperref}
\hypersetup{
    colorlinks,
    citecolor=blue,
    filecolor=black,
    linkcolor=blue,
    urlcolor=blue,
}

\usepackage{amsmath,amsthm,amssymb,mathtools,bm}

\usepackage[margin=1in]{geometry}


\numberwithin{equation}{section}

\newlength\tindent
\setlength{\tindent}{\parindent}
\setlength{\parindent}{0pt}
\renewcommand{\indent}{\hspace*{\tindent}}

\allowdisplaybreaks[1]


\newcommand{\makered}[1]{{\color{red}#1}}
\newcommand{\makeblue}[1]{{\color{blue}#1}}

\newcommand{\vect}[1]{\mathbf{{#1}}}
\newcommand{\mat}[1]{\mathbf{{#1}}}

\title{Optimal Test Inner Product for Linear Advection}
\author{Philip Zwanenburg}

\begin{document}
\maketitle

%%%%%%%%%%%%%%%%%%%%%%%%%%%%%%%%%%%%
%%% Begin Modifiable parameters. %%%
%%%%%%%%%%%%%%%%%%%%%%%%%%%%%%%%%%%%

\begin{sagesilent}
var('a,b,c')
assume(a>=0)
assume(b>=0)
assume(c>=0)

c = a
\end{sagesilent}

%%%%%%%%%%%%%%%%%%%%%%%%%%%%%%%%%%%%
%%% End Modifiable parameters.   %%%
%%%%%%%%%%%%%%%%%%%%%%%%%%%%%%%%%%%%

\begin{sagesilent}
var('h')
var('r')
var('i,j,k,l')

assume(i,j,k,l,'integer')
\end{sagesilent}

%%%%%%%%%%%%%%%%%%%%%%%%%%%%%%%%%%%%%%%%%%%%%%%%%%%%%%%%%%%%%%%%%%%%%%%%%%%%%%%%%%%%%%%%%%%%%%%%%%%%
\section{Test Functions}

It has been noted that under specific norms, the optimal test functions are given by polynomials one degree higher than
that of the corresponding solution basis. Below, the analytical expressions for the test functions are obtained and they
are then substituted into the bilinear form to determine the associated induced norms. For the 1D case under
consideration, with volumes numbered from $1$ to $n$ and face nodes numbered from $0$ to $n$, we choose to work with the
general test norm
\begin{align*}
(w,v) = \sum_{i=1}^{n}
\int_{-1}^{1} \frac{2}{h} w_i^{'}v_i^{'}
+ a w_i(1)v_i(1)
+ b(w_i(1)-w_{i+1}(-1))(v_i(1)-v_{i+1}(-1))
+ c w_{i+1}(-1)v_{i+1}(-1).
\end{align*}

Optimal test functions for a given basis function are then found by solving the following system of equations
\begin{align} \label{eq:eq_for_v}
(w,v) = b(w,\phi)\ \forall w \in V
\end{align}

where $\phi$ denotes a basis function from the trial space.

%%%%%%%%%%%%%%%%%%%%%%%%%%%%%%%%%%%%%%%%%%%%%%%%%%%%%%%%%%%%%%%%%%%%%%%%%%%%%%%%%%%%%%%%%%%%%%%%%%%%
\subsection{Volume Test Functions}

It was observed numerically, when using the Legendre polynomials as volume solution basis functions, that all associated
test functions except that of the constant basis are zero at both edges of the reference element. Further, they all
satisfy~\eqref{eq:eq_for_v} exactly when the test space is one order higher than the solution space. Consequently, only
the $p0$ test function need be computed. Noting that the $p0$ test function is linear, represented as
\begin{align*}
v_{{\phi}_{i,0}} = a_0 + a_1 r,
\end{align*}

it can be determined by solving the following equation for the coefficients, for the general form of the test norm
\begin{align*}
\int_{-1}^{1} \frac{2}{h} w_i^{'} v_i^{'} dr + (a+b)(w_i(1)v_i(1))+(b+c)(w_i(-1)v_i(-1))
= 
\int_{-1}^{1} -w_i^{'} \phi dr,\ \forall w_i \in \mathcal{P}^1,
\end{align*}

where $\phi_0 = \frac{1}{\sqrt{2}}$, and $\mathcal{P}^p$ is the space of all polynomials of degree less than order
equal to $p$. Choosing $w_i = 1$ and $w_i = r$, we obtain the following equalities
\begin{align*}
0 + (a+b)((1)v_i(1))+(b+c)((1)v_i(-1)) 
& =
0, \\
\int_{-1}^{1} \frac{2}{h} (1) v_i^{'} dr + (a+b)((1)v_i(1))+(b+c)((-1)v_i(-1))
& = 
\int_{-1}^{1} (-1) \frac{1}{\sqrt{2}} dr.
\end{align*}

Substituting the general expression for $v_{{\phi}_{i,0}}$, we obtain the coefficients by solving the following linear
system
\begin{sagesilent}
n = 2
AV = matrix(SR,n,n,[[a+2*b+c,a-c],
                    [a-c,4/h+a+2*b+c],
                   ])
BV = matrix(SR,n,1,[0,-2/sqrt(2)])

VVCoef = AV\BV
for i in range(0,n):
    VVCoef[i] = (VVCoef[i]).apply_map(lambda x: x.rational_simplify().expand().full_simplify())
\end{sagesilent}

\[
\mat{A} \hat{\vect{v}} = \vect{b}
\]

where
\[
\mat{A} = \sage{AV},\ \vect{b} = \sage{BV},\ \text{and}\ \hat{\vect{v}} = [a_0,a_1]^T.
\]

The result is
\begin{sagesilent}
VPhi0 = VVCoef[0]+VVCoef[1]*r
\end{sagesilent}

\[
v_{{\phi}_{i,0}} = \sage{VPhi0}.
\]

%%%%%%%%%%%%%%%%%%%%%%%%%%%%%%%%%%%%%%%%%%%%%%%%%%%%%%%%%%%%%%%%%%%%%%%%%%%%%%%%%%%%%%%%%%%%%%%%%%%%
\subsection{Face Test Functions}

The linear test functions on either side of the 1D face (point) take the exact form:
\begin{align*}
& v_{\hat{\phi}_i}^l = a_0^l + a_1^l r,\\
& v_{\hat{\phi}_i}^r = a_0^r + a_1^r r.
\end{align*}

The coefficients can be computed by solving the following linear system

\begin{sagesilent}
n = 4
AF = matrix(SR,n,n,[[a+2*b+c,a-c,-b,b],
                    [a-c,4/h+a+2*b+c,-b,b],
                    [-b,-b,a+2*b+c,a-c],
                    [b,b,a-c,4/h+a+2*b+c]
                   ])
BF = matrix(SR,n,1,[1,1,-1,1])

VFCoef = AF\BF
for i in range(0,n):
    VFCoef[i] = (VFCoef[i]).apply_map(lambda x: x.rational_simplify().expand().full_simplify())
\end{sagesilent}

\[
\mat{A} \hat{\vect{f}} = \vect{b}
\]

where
\[
\mat{A} = \sage{AF},\ \vect{b} = \sage{BF},\ \text{and}\ \hat{\vect{f}} = [a_0^l,a_1^l,a_0^r,a_1^r]^T.
\]

The result is
\begin{sagesilent}
VPhiHatL = VFCoef[0]+VFCoef[1]*r
VPhiHatR = VFCoef[2]+VFCoef[3]*r
\end{sagesilent}

\begin{align*}
& v_{\hat{\phi}_i}^l = \sage{VPhiHatL}, \\
& v_{\hat{\phi}_i}^r = \sage{VPhiHatR}.
\end{align*}



%%%%%%%%%%%%%%%%%%%%%%%%%%%%%%%%%%%%%%%%%%%%%%%%%%%%%%%%%%%%%%%%%%%%%%%%%%%%%%%%%%%%%%%%%%%%%%%%%%%%
%\section*{References}

%\bibliographystyle{elsarticle-num}
%\bibliography{../code.bib}

\end{document}
