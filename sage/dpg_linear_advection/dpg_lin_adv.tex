\documentclass{article}

% To compile the pdf, execute the following in the terminal:
% $ pdflatex ${file_name}.tex
% $ bibtex ${file_name}
% $ sage ${file_name}.sagetex.sage
% $ pdflatex ${file_name}.tex

\usepackage{sagetex}
\setlength{\sagetexindent}{10ex}

\usepackage{hyperref}
\hypersetup{
    colorlinks,
    citecolor=blue,
    filecolor=black,
    linkcolor=blue,
    urlcolor=blue,
}

\usepackage{amsmath,amsthm,amssymb,mathtools,bm}
\mathtoolsset{showonlyrefs} % Only number equations which are referred to in text.

\usepackage[margin=1in]{geometry}


\numberwithin{equation}{section}

\newlength\tindent
\setlength{\tindent}{\parindent}
\setlength{\parindent}{0pt}
\renewcommand{\indent}{\hspace*{\tindent}}

\allowdisplaybreaks[1]


\newcommand{\makered}[1]{{\color{red}#1}}
\newcommand{\makeblue}[1]{{\color{blue}#1}}

\newcommand{\varg}[1]{\mathit{\bm{{#1}}}} % 'g'roup  'var'iable
\newcommand{\vect}[1]{\mathbf{{#1}}}
\newcommand{\mat}[1]{\mathbf{{#1}}}

\usepackage{stmaryrd} % For jump operator brackets
\newcommand{\jump}[1]{{\llbracket #1\rrbracket}}

\title{Optimal Test Inner Product for Linear Advection}
\author{Philip Zwanenburg}

\begin{document}
\maketitle

%%%%%%%%%%%%%%%%%%%%%%%%%%%%%%%%%%%%
%%% Begin Modifiable parameters. %%%
%%%%%%%%%%%%%%%%%%%%%%%%%%%%%%%%%%%%

\begin{sagesilent}
var('a,b,c')
assume(a>=0)
assume(b>=0)
assume(c>=0)

#c = a;
b = 0; c = 0;
\end{sagesilent}

%%%%%%%%%%%%%%%%%%%%%%%%%%%%%%%%%%%%
%%% End Modifiable parameters.   %%%
%%%%%%%%%%%%%%%%%%%%%%%%%%%%%%%%%%%%

\begin{sagesilent}
var('h')
var('r')
var('i,j,k,l')

assume(i,j,k,l,'integer')
\end{sagesilent}

%%%%%%%%%%%%%%%%%%%%%%%%%%%%%%%%%%%%%%%%%%%%%%%%%%%%%%%%%%%%%%%%%%%%%%%%%%%%%%%%%%%%%%%%%%%%%%%%%%%%
\section{Test Functions}

For the 1D case under consideration, with volumes numbered from $1$ to $n$ and face nodes numbered from $0$ to $n$, the
bilinear form associated with the steady linear advection equation (assuming a unit advection velocity) is given by
\begin{align} \label{eq:bilinear_adv}
b(v,\varg{u}) 
& = \sum_{i=1}^n -\int_{-1}^{1} v_i^{'} u_i dr - v_i(-1)f_{i-1}+v_{i}(1))f_i\ \forall v
%\\
%& = \sum_{i=1}^n -\int_{-1}^{1} v_i^{'} u_i dr + \sum_{i=0}^n \jump{v} f_i,\ \forall v.
\end{align}

Above, $\varg{u} \coloneqq (u,f)$ is the group variable for the solution and trace flux components, and $v$ is a test
function and where all quantities have been transferred to the reference volume and face.
%The \textit{jump} operator, $\jump{v} = v^- - v^+ $, was also introduced with ``$-$'' and ``$+$'' referring to the
%volumes adjacent to the face with the normal vector pointing outwards/inwards, respectively, and with the additional
%specification of $v^+ = \pm v^-$ on inflow/outflow boundaries, respectively. Following the motivation of pursuing norms
\\~

It has been noted that under specific norms, the optimal test functions are given by polynomials one degree higher than
that of the corresponding solution basis~\cite[Section \makeblue{3C}]{Demkowicz2011}. Below, the analytical expressions
for the test functions are obtained and they are then substituted into the bilinear form to determine the associated
    induced norms. We choose to work with the general test norm
\begin{align}
(w,v) = \sum_{i=1}^{n}
\int_{-1}^{1} \frac{2}{h} w_i^{'}v_i^{'} dr
+ a w_i(1)v_i(1)
+ c w_{i}(-1)v_{i}(-1)
+ \sum_{i=0}^{n}
+ b(w_i(1)-w_{i+1}(-1))(v_i(1)-v_{i+1}(-1))
\end{align}

where it is to be assumed that quantities not present in the domain are omitted in the last summation. Note that this test
norm recovers that of Demkowicz et al. from the first DPG paper~\cite[Section \makeblue{2C}]{Demkowicz2011} when
selecting the parameters in the norm above as $a = \alpha_i, b = 0, c = 0$. In this report we have taken the values of
the parameters to be constant for all elements and equal to
\begin{align}
[a,b,c] = [\sage{a},\sage{b},\sage{c}].
\end{align}

Optimal test functions for a given basis function are then found by solving the following system of equations
\begin{align} \label{eq:eq_for_v}
(w,v) = b(w,\phi)\ \forall w \in V
\end{align}

where $\phi$ denotes a basis function from the trial space.

%%%%%%%%%%%%%%%%%%%%%%%%%%%%%%%%%%%%%%%%%%%%%%%%%%%%%%%%%%%%%%%%%%%%%%%%%%%%%%%%%%%%%%%%%%%%%%%%%%%%
\subsection{Volume Test Functions}

It can be observed, when using the Legendre polynomials as volume solution basis functions, that all associated
test functions except that of the constant basis are zero at both edges of the reference element~\cite[Section
5.1]{BuiThanh2013}. Further, they all satisfy~\eqref{eq:eq_for_v} exactly when the test space is one order higher than
the solution space. Consequently, only the $p_0$ test function needs to be computed. Noting that the $p_0$ test function is
linear, represented as
\begin{align}
v_{{\phi}_{i,0}} = a_0 + a_1 r,
\end{align}

it can be determined by solving the following equation for the coefficients, for the general form of the test norm
\begin{align}
\int_{-1}^{1} \frac{2}{h} w_i^{'} v_i^{'} dr + (a+b)(w_i(1)v_i(1))+(b+c)(w_i(-1)v_i(-1))
= 
\int_{-1}^{1} -w_i^{'} \phi dr,\ \forall w_i \in \mathcal{P}^1,
\end{align}

where $\phi_0 = \frac{1}{\sqrt{2}}$, and $\mathcal{P}^p$ is the space of all polynomials of degree less than order
equal to $p$. Choosing $w_i = 1$ and $w_i = r$, we obtain the following equalities
\begin{align}
0 + (a+b)((1)v_i(1))+(b+c)((1)v_i(-1)) 
& =
0, \\
\int_{-1}^{1} \frac{2}{h} (1) v_i^{'} dr + (a+b)((1)v_i(1))+(b+c)((-1)v_i(-1))
& = 
\int_{-1}^{1} (-1) \frac{1}{\sqrt{2}} dr.
\end{align}

Substituting the general expression for $v_{{\phi}_{i,0}}$, we obtain the coefficients by solving the following linear
system
\begin{sagesilent}
n = 2
AV = matrix(SR,n,n,[[a+2*b+c,a-c],
                    [a-c,4/h+a+2*b+c],
                   ])
BV = matrix(SR,n,1,[0,-2/sqrt(2)])

VVCoef = AV\BV
for i in range(0,n):
    VVCoef[i] = (VVCoef[i]).apply_map(lambda x: x.rational_simplify().expand().full_simplify())
\end{sagesilent}

\begin{align}
\mat{A} \hat{\vect{v}} = \vect{b}
\end{align}

where
\begin{align}
\mat{A} = \sage{AV},\ \vect{b} = \sage{BV},\ \text{and}\ \hat{\vect{v}} = [a_0,a_1]^T.
\end{align}

The result is
\begin{sagesilent}
VPhi0 = (VVCoef[0]+VVCoef[1]*r)[0]
\end{sagesilent}

\begin{align} \label{eq:v_opt_vol_p0}
v_{{\phi}_{i,0}} = \sage{VPhi0}.
\end{align}

%%%%%%%%%%%%%%%%%%%%%%%%%%%%%%%%%%%%%%%%%%%%%%%%%%%%%%%%%%%%%%%%%%%%%%%%%%%%%%%%%%%%%%%%%%%%%%%%%%%%
\subsection{Face Test Functions}

The linear test functions on either side of the 1D face (point) take the exact form:
\begin{align}
& v_{\hat{\phi}_i}^l = a_0^l + a_1^l r,\\
& v_{\hat{\phi}_i}^r = a_0^r + a_1^r r.
\end{align}

The coefficients can be computed by solving the following linear system

\begin{sagesilent}
n = 4
AF = matrix(SR,n,n,[[a+2*b+c,a-c,-b,b],
                    [a-c,4/h+a+2*b+c,-b,b],
                    [-b,-b,a+2*b+c,a-c],
                    [b,b,a-c,4/h+a+2*b+c]
                   ])
BF = matrix(SR,n,1,[1,1,-1,1])

VFCoef = AF\BF
for i in range(0,n):
    VFCoef[i] = (VFCoef[i]).apply_map(lambda x: x.rational_simplify().expand().full_simplify())
\end{sagesilent}

\begin{align}
\mat{A} \hat{\vect{f}} = \vect{b}
\end{align}

where
\begin{align}
\mat{A} = \sage{AF},\ \vect{b} = \sage{BF},\ \text{and}\ \hat{\vect{f}} = [a_0^l,a_1^l,a_0^r,a_1^r]^T.
\end{align}

The result is
\begin{sagesilent}
VPhiHatL = (VFCoef[0]+VFCoef[1]*r)[0]
VPhiHatR = (VFCoef[2]+VFCoef[3]*r)[0]
\end{sagesilent}

\begin{align}
& v_{\hat{\phi}_i}^l = \sage{VPhiHatL}, \\
& v_{\hat{\phi}_i}^r = \sage{VPhiHatR}.
\end{align}


%%%%%%%%%%%%%%%%%%%%%%%%%%%%%%%%%%%%%%%%%%%%%%%%%%%%%%%%%%%%%%%%%%%%%%%%%%%%%%%%%%%%%%%%%%%%%%%%%%%%
\section{Implied Energy Norm}
% Note: Dividing quantities by BasisU0 such that we include its contribution in u_{i,0} = BasisU0*coef_u_{i,0}.
\begin{sagesilent}
dVPhi0 = diff(VPhi0,r)
BasisU0 = sqrt(1/2)
dVPhi0U = (dVPhi0/BasisU0).rational_simplify()
\end{sagesilent}

The implied energy norm is obtained by substituting the optimal test functions into the bilinear form as discussed by 
Demkowicz et al.~\cite[eq. \makeblue{(2.8)} and Proposition \makeblue{2.2}]{Demkowicz2011}. Substituting the $p_0$
optimal test function,~\eqref{eq:v_opt_vol_p0}, into the bilinear form,~\eqref{eq:bilinear_adv}, after multiplication by
the solution coefficients
\begin{align}
b(v_{u_{0}},\varg{u}) 
& = \sum_{i=1}^n
\int_{-1}^{1} \sage{(-dVPhi0U).rational_simplify()} u_{i,0}u_i dr
- \left( \sage{(VPhi0.subs({r:-1})/BasisU0).rational_simplify()} \right) u_{i,0}f_{i-1}
+ \left( \sage{(VPhi0.subs({r:1})/BasisU0).rational_simplify()} \right) u_{i,0}f_i
\end{align}

where $u_{i,0} = \frac{1}{h} \int_{x_{i-1}}^{x_i} u dx = \frac{1}{2} \int_{-1}^{1} u dr$ denotes the average of
$u_i$ in volume $i$. Noting the property discussed above of all other integrated Legendre polynomials taking values of
zero at $r = \pm 1$, we have
\begin{align}
b(v_{u_{j>0}},\varg{u}) 
& = \sum_{i=1}^n
\int_{-1}^{1} \frac{h}{2} (u_i-u_{i,0})u_i dr.
\end{align}

Summing the two contributions
\begin{align}
b(v_{u},\varg{u}) 
& = \sum_{i=1}^n
\int_{-1}^{1} \frac{h}{2} u_i^2 dr
- \left( \sage{(VPhi0.subs({r:-1})/BasisU0).rational_simplify()} \right) u_{i,0}f_{i-1}
+ \left( \sage{(VPhi0.subs({r:1})/BasisU0).rational_simplify()} \right) u_{i,0}f_i.
\end{align}
\makered{Needs extra volume term for cases where it is not zero.}\\~

Considering the optimal test functions for the fluxes,
\begin{sagesilent}
var('fl,fr')
MultFlr1 = (VPhiHatL.subs({r:-1})).rational_simplify()
MultFrr  = (VPhiHatL.subs({r:1})).rational_simplify()
MultFll  = (VPhiHatR.subs({r:-1})).rational_simplify()
MultFlr2 = (VPhiHatR.subs({r:1})).rational_simplify()
\end{sagesilent}

\begin{align}
b(v_{f},\varg{u}) 
= \sum_{i=1}^n
& \int_{-1}^{1} \left( \sage{(-diff(VPhiHatL,r)).rational_simplify()} \right) f_{i}u_{i} dr
- \left( \sage{MultFlr1} \right) f_{i} f_{i-1}
+ \left( \sage{MultFrr}  \right) f_{i}^2 \\
+ & \int_{-1}^{1} \left( \sage{(-diff(VPhiHatR,r)).rational_simplify()} \right) f_{i-1}u_{i} dr
- \left( \sage{MultFll} \right) f_{i-1}^2
+ \left( \sage{MultFlr2}  \right) f_{i-1}f_{i}
\end{align}

% Use .factor() to combine expanded terms.


%%%%%%%%%%%%%%%%%%%%%%%%%%%%%%%%%%%%%%%%%%%%%%%%%%%%%%%%%%%%%%%%%%%%%%%%%%%%%%%%%%%%%%%%%%%%%%%%%%%%
%\section*{References}

\bibliographystyle{elsarticle-num}
\bibliography{../../PhD_Thesis.bib}

\end{document}
