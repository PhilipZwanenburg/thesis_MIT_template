\section{Contributions}
While the ultimate goal of the thesis work presented here
was related to the last of the items listed in~\autoref{sec:intro_background}, namely the investigation of
the advatages of of introducing the required stabilization through the variation
of the test space as opposed to using a numerical flux,
several difficulties encountered while working towards the necessary
framework ultimately ended up forming a dominant part of the novel contributions
which resulted from this research effort.

As emphasized in the motivation,~\autoref{sec:intro_motivation}, critical to the
success of high-order methods is that that optimal
convergence rates be obtained.
During the course of the verification of the methods implemented for
the thesis work several issues resulting in the loss of optimal convergence were
identified in relation to the treatment of curved geometry.
Specifically, we have proven that the usage of high-order meshes violating a
mesh-dependent discrete curvature constraint results in a loss of optimal
convergence.
In parallel with this investigation, we discovered a generalized
constraint on the necessary polynomial extension of curved boundary geometry lifted
to the volume.
This result recovers all successful existing options previously
presented in the two-dimensional context and provides novel generalizations to
three dimensions for simplicial elements.
Finally, we have also uncovered a mechanism responsible for
the loss of half of an order of convergence, as compared to the optimal
convergence rate, for PDEs employing normal-dependent boundary conditions on
curved boundaries, extending previous results.


\makered{Add necessary modifications.} In the context of the investigation of
methods with improved stabilization, our success has been limited. Beginning
with the theoretical investigation of the simplest model PDE for hyperbolic
conservation laws, the linear advection equation, the determination of optimal
spaces for proper stabilization led to the conclusion that this goal was not
achievable when the dimension of the problem was greater than one.

Following up with the numerical implementations of both the DPG and OPG methods, it quickly
became clear that significant challenges existed when considering even marginal extensions
of the work presented in the literature. In particular:
\begin{itemize}
\item the discrete spaces used to motivate the superiority of the methods
  required that the globally coupled unknowns correspond to a polynomial
  representation one higher than that used for HDG and HDPG methods,
  resulting in the Petrov-Galerkin methods considered here having global systems corresponding
  to those of alternative methods of one higher degree;
\item the specification of suitable boundary conditions was found to be problematic
  in all but the most straightforward contexts;
\item \makered{the use of the methods based on the solution of linearized PDEs
  resulted in a stalled iterative procedure when strong nonlinearities were
  present and a physical constraint (such as positivity of density and pressure)
  was required for the solution.}
  % Not yet implemented: try the viscous burgers example from Jesse (p. 147)
  % comparing the linear and nonlinear test space and hope for improvement.
\end{itemize}

While we propose extensions allowing for the treatment of several of these
issues, our attempts have generally resulted in the loss of advanteous
properties of the methods. For example, the imposition of general boundary
conditions was performed using the same procedure as that employed for the DG
method, resulting generally in a loss of symmetry of the global system matrix
present for the DPG method in simpler contexts.
Regarding the implementation of DPG applied to the nonlinear PDEs, it was found
that Hessian terms required for the exact linearization to allow for quadratic
convergence in the Newton method resulted in an extremely high computational
cost as compared to the method as applied to the linearized PDEs.

Perhaps more importantly, the alledged superiority of the Petrov-Galerkin
methods was immediately put into question as a result of the required increased
degree in the discrete polynomial spaces. The last contribution made here was
thus to investigate whether the improved stability properties of the methods
could ever justify their increased cost based on the solution of challenging
test cases for DG methods.

Finally, while not at all related to research goals for the thesis as presented
here, we would like to note that a significant amount of time was also devoted
to a generalized demonstration of the equivalence between the Energy Stable Flux
Reconstruction (ESFR) schemes~\cite{Vincent2011,Williams2014a} and a modally
filtered DG scheme~\cite{Zwanenburg2016} in 3D curvilinear domains for both
tensor-product and simplex elements.

\makegreen{Goal:Try DPG on Navier-Stokes case with shock and show better
  stabilization. Alternatively, used DPG for coarse mesh and use as initial
  solution for DG showing that the ball of convergence is reached more quickly.}
