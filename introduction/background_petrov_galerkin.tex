\subsection{Petrov-Galerkin Approaches for Stabilization}
\label{sec:background_pg_stabilization}

The Galerkin method was originally developed for the solution of PDEs in
structural mechanics in which it can be shown to provide optimal results based
on the implicit relationship betwen the resulting variational formulation and
the minimization of an energy functional.
In the setting of convection-dominated or purely hyperbolic PDEs, the optimal
convergence result is lost and it is now well understood that additional
stabilization is required.
In the linear setting, the degradation in solution accuracy,
generally manifested through the development of unphysical oscillations
throughout the domain, can be formally explained by appealing the approximation
result provided by Babu\v{s}ka's theorem~\cite[Theorem \makeblue{2.2}]{Babuska1971}
which states that the Galerkin error for a well-posed problem satisfies
\begin{align}
||u-u_h||_{\fspace{U}} \le \left( 1 + \frac{M}{\gamma_h} \right) \inf_{w_h \in \fspace{U}_h} ||u-w_h||_{\fspace{U}},
\end{align}

where $M$ and $\gamma_h$ are the continuity and discrete inf-sup constants. The
use of improper stabilization can be directly linked to the vanishing of the
discrete inf-sup constant. This result is not directly generalizable to
the context of the nonlinear PDEs of interest for CFD, but it is still expected
that employing a formulation in which the discrete inf-sup constant is increased
should result in improved stability in a wide range of contexts.

Through the recently popularized interpretation of numerical flux induced
stabilization associated with the DG-type methods through a suitable
penalization of interface jumps~\cite{Brezzi2004,DiPietro2011} it is possible to
derived stability estimates providing explicit representations for the
continuity and inf-sup constants. However, it is not generally possible to
obtain optimal stabilization in this setting (maximizing the discrete inf-sup
constant) nor does the norm used to measure the solution,
$||\cdot||_{\fspace{U}}$, necessarily correspond to that which is most
desirable (notably the $\fspace{L}^2$ norm).

Several methods introducing the
stabilization through the variation of the test space, resulting in a
Petrov-Galerkin framework, have been shown to have the flexibility to meet these
objectives.
The first attempt at the introduction of stabilization through the variation of
the test space was provided in the context of continuous finite element
solutions through the use of residual-based stabilization in the form of the
Streamline Upwind Petrov Galerkin (SUPG) formulation~\cite{Brooks1982}.
It was subsequently shown by the same author that the SUPG formulation was in
essence complementing the solution space with information related to the
fine-scale (unresolvable by the current mesh) Green's function in the context of
the variational multiscale method~\cite{Hughes1998,Hughes2007}.
Approaching the problem from the functional analysis setting discussed above,
the discontinuous Petrov-Galerkin method with optimal test functions was
introduced where the test space is chosen specifically to achieve the supremum
in the discrete inf-sup condition~\cite{Demkowicz2010,Demkowicz2017}; in
essence, the test space is chosen for its good stability properties as opposed
to good approximation properties required for the trial space.
In the CFD setting,
a further emphasized advantage is that the formulation naturally precludes the
need for the formulation of numerical fluxes, requiring the selection of a norm
for the test space instead, which has the potential of being more naturally
selected based on the terms present in the variational form.
A hybridized discontinuous Petrov-Galerkin method~\cite{Moro2012} has also been
proposed as a blend between the HDG and DPG schemes with the goal of retaining
the minimally number of globally coupled DOF while retaining the optimal
stability properties of the DPG scheme. Several subsequent investigations have
focused on attempting to find optimal test norms defined in the sense of the
computed solution being given by the $\fspace{L}^2$ projection of the exact
solution. The investigation was initially undertaken by Bui-Thanh et
al.~\cite{BuiThanh2013} resulting in an impractical method as a result of the
test functions having global support, but eventually reformulated by Brunken et
al.~\cite{Brunken2018} in what will be referred to as the optimal trial
Petrov-Galerkin (OPG) method such that a practical method to achieve this goal was
obtained.

The improved stability of the Petrov-Galerkin methods thus has the potential to reduce the need
for additional ad-hoc approaches where oscillations are popularly suppressed
either through the addition of artificial viscosity or by limiting the solution
using a modal filtering approach. By minimizing modifications to the computed
solution, it is then expected that convergence to the nonlinear solution would
be more robust on underresolved meshes and also that adaptation mechanisms
relying on a posteriori error indicators would have a reduced tendency to refine
the discrete space in incorrect regions as a result of the introduced solution regularization.
The DPG and HDPG methods have both been shown to be able to converge in cases
where DG-type methods have failed, generally in the presence of the
solution having large gradients when no additional stabilization is added.
However, results have generally shown that DPG solutions possess qualitatively
similar oscillations to those present when using the DG-type methods when the
element P\'eclet number is on the order of $\mathcal{O}(10)$, only one order
of magnitude greater than that leading to oscillations for the DG methods.
Further investigation and comparison is thus required in order to establish whether
the added cost is justified.