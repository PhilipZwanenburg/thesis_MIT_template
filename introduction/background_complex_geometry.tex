\subsection{Treatment of Complex Geometry}
\label{sec:background_complex_geom}

The proper treatment of complex geometry in high-order finite element methods has been
shown to be crucial. In the seminal work on the topic in the context of elliptic
PDEs, it was proven that curved geometry had to be represented
isoparametrically, i.e. in a polynomial space having the same degree as the
solution, with specific constraints on the polynomial representation employed
for optimal convergence to be obtainable~\cite[Theorem
\makeblue{5}]{Ciarlet1972}.
While errors due to improper geometry representation may only begin to manifest
themselves at very fine levels of solution resolution, it is precisely the goal
of high-order methods to achieve these levels of accuracy. Further, if these
geometric errors result in decreasing convergence rates of high-order methods,
then the increase in the computational complexity would be incurred with no
additional benefit.

As high-order meshes are commonly generated from the degree elevation of
initially linear meshes, the manner in which the projection of the initially
straight-sided mesh to the curvilinear domain is performed is of critical importance.
In the early years of the application of finite element methods to problems
in domains having curved boundaries, many seemingly disjointed strategies
were proposed to achieve the correct polynomial geometry representation satisfying the
necessary constraints. Assuming suitable placement of geometry interpolation
nodes along the curved boundary, the transfer of the curved face representation
to the volume geometry nodes was achieved using transfinite blending function
interpolation, first proposed by Gordon et al. for tensor-product
elements~\cite{Gordon1973} and subsequently generalized to simplex
elements in both
two~\cite{Nielson1979,Haber1981,Szabo1991,Lacombe1988,Dey1997,Xie2013} and
three~\cite{Lenoir1986} dimensions.
Assuming that a Lagrange polynomial description of the geometry is being used
and that corner nodes are located on the exact curved boundary, the process
proceeds by sequentially projecting straight edge and face nodes to the curved
geometry followed by the application of a blending operation which appropriately
displaces the volume nodes (those not on the domain boundary). Recently, an additional
constraint related to the discrete curvature of the meshed domain was shown to
be necessary for optimal convergence \makered{cite Zwanenburg - Discrete
  Curvature}. In the same article, guidelines were provided concerning the
correct placement of face geometry nodes and a unification of two-dimensional
blending function interpolations and generalization to the three-dimensional
case was presented.

Perhaps surprisingly, numerical results for the Euler equations showed that a
superparametric geometry representation, with polynomial degree one higher than
the solution, was \emph{required} for optimal convergence only when the solution
was represented by a polynomial basis having degree $p = 1$~\cite{Bassi1997};
this numerical result has since been extended to show that the superparametric
geometry representation is in fact required for all polynomial
degrees~\cite{Zwanenburg2017}.
Initially, this phenomenon was explained using the argument that the
that low-order geometry representation results in deterioration of
solution quality as the order of the scheme is increased due to rarefaction
waves being formed at vertices of polygonal mesh
surfaces~\cite{Krivodonova2006}. However, a thorough analysis of the problem was
recently performed in which it was demonstrated that the problem occurs in all
instances in which a boundary condition is used which dependends on a normal
vector computed using the isoparametric geometry representation \makered{cite
  Zwanenburg - Necessity superparametric}. A result of particular interest
arising from this study was the demonstration that the use of exact normal
vectors in combination with isoparametric volume metric terms does not remedy
the problem due to the violation of discrete metric identities resulting in a
high-order conservation error of the same magnitude as that introduced by the
isoparametric normal vectors. This represents the high-order analogue
of the violation of free-stream preservation, which has been shown to be
avoidable by computing metrics according to an elegant curl-formulation by
Kopriva~\cite{Kopriva2006}.

It is important to emphasize that the discussion above assumed that a valid linear
mesh could initially be generated for the complex geometry to be modelled in the
simulation. In fact, it has recently been estimated that approximately 80\% of
overall analysis time in the aerospace industry is devoted to (linear) mesh
generation~\cite{Hughes2005}, resulting in serious challenges when attempting to
interface between simulated results and the
Computer Aided Design (CAD) model, for example. This motivated the formulation
of isogeometric analysis (IGA) where the solution is represented in the same basis as
the CAD geometry (non-polynomial), allowing for perfect geometric representation at any level of
mesh refinement and seamless interfacing with the CAD model~\cite{Hughes2005}.
While the competitiveness of this new approach with existing methods has been
demonstrated in numerous academic benchmark test cases in both structural and fluid
mechanics, its general applicability to test cases having sufficient geometric
complexity to be relevant to the CFD industry is still in question. Further, the
extension of the approximation theory results from the polynomial context discussed
above to the most popular IGA setting employing non-uniform rational spline as
basis functions is still in its infancy.