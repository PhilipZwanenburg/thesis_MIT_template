\section{Motivation}
\label{sec:intro_motivation}

The use of computational fluid dynamics (CFD) tools for the numerical analysis
of fluid flows has significantly reduced costs associated with aerodynamic
design over the past several decades.
As computing systems have become
increasingly powerful, there has been a corresponding advance in the equations
employed for flow simulation (initially beginning with potential equations and
now generally using the compressible Navier-Stokes equations) as well as in the resolution of
complex flow phenomena.
Finite volume methods currently represent the industry
and, to a great extent, academic standard for the solution of partial
differential equations (PDEs) in the aerospace community.
This is in large part
due to their robustness in the presence of steep gradients in the flow as well
as their ability to model geometrically complex objects as a result of the
possibly unstructured nature of the volumes.
However, finite volume schemes are
inherently second order accurate, making their usage inefficient when flow
solutions have high regularity.

As a result of significant interest from the research community, high-order
methods are thus now being pursued with the goal of achieving solution
convergence at higher than second order rates and of employing fewer degrees of
freedom (DOF) to represent solutions with comparable error magnitudes.
In the case of the solution having unlimited regularity, the ideal high-order
strategy is to employ spectral methods for their representation due to the
exponential convergence properties and extremely efficient solution algorithms.
However, these methods are unsuitable when the domain under consideration has
complex geometrical features or when the regularity of the solution varies
throughout the domain.
The natural path of development has thus been towards the use of
pseudo-spectral methods, which can be thought of in terms of employing a local
spectral method within each volume contributing to the tessellation of the
considered domain.
The most popular and convenient approach for high-order
solution representation is through the use piecewise polynomial basis functions
having limited support.
Within this framework, it is expected that the
combination of mesh refinement level adaptation ($h$-adaptation) and polynomial
degree adaptation ($p$-adaptation) can be used to obtain a very efficient
representation of the solution, and reduce the computational cost of obtaining
the solution as compared to low order methods.

Of course, it must be noted that significant challenges remain in the setting of
high-order methods.
As the advantage of high-order methods is necessarily linked to their
high-order convergence rates, of primary importance is that methods be
designed such that expected convergence rates from the polynomial approximation
theory be achievable; this rate of convergence is termed the optimal rate.
Further, all relevant aspects of the discretization must be
considered such that the optimal rates are observed in practice.
Another pressing challenge stems from the fact that high-order methods have less
numerical diffusion as compared to low-order methods, leading in many cases to
the requirement for the addition of additional stabilization to that
provided by the scheme.
This stabilization is commonly introduced either through
the use of a limiter or by adding an artificial dissipation term to
discretization.
Both methods result in a modification of the computed solution
as compared to that which is associated with the PDE and the investigation of
methods naturally introducing the physically correct stabilization are
consequently of great importance as well.
