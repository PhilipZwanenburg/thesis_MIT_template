\subsection{Computational Complexity}
\label{sec:background_comp_comp}

Significant progress has been made with regard to the DG method's computational
complexity, notably through the exploitation of sum factorization techniques,
originally proposed by Orszag~\cite{Orszag1980} and now employed in the
general elements using collapsed tensor-product spaces by Karniadakis et
al.~\cite{Karniadakis1999}. When explicit methods are used, the sum
factorization technique reduces the growth in computational complexity from
$O(N^{2d})$ to $O(N^{d+1})$ where $d$ is the dimension of the problem and $N =
P+1$ where $P$ is the maximal polynomial degree of the basis functions used to
represent the solution; savings for implicit formulations are even greater.

Alternatively, in the case of using implicit methods, the appropriate
decomposition of element bases into corner,
edge, face, and volume modes allows for the possibility of statically condensing out
the volume modes, significantly reducing the growth rate of globally coupled DOF
as the degree of the solution is increased in global matrix inversion stages.
In practice, this choice of polynomial basis representation is rarely used due
to the advantages associated with orthonormal, interpolatory or everywhere
positive basis functions and this reduction in DOF is not possible. This
provided the motivation for the development of what are termed
hybridizable methods where schemes were specifically designed such that global
coupling of the solution occurs only through unknowns defined on element faces,
such that the size of the global system scales in proportion to the number of
face unknowns, exactly corresponding to the static condensation of volume
unknowns.

The hybridizable discontinuous Galerkin (HDG) method was initially
introduced in the context of elliptic equations~\cite{Cockburn2009}, but was
quickly generalized to more complex partial differential equations, notably
to the compressible Navier-Stokes equations~\cite{Peraire2010}.
More recently, a generalized framework called the Hybrid High-Order method which
is able to recover specific HDG formulations with superoptimal convergence for
elliptic problems and which supports general polytopal elements was
proposed~\cite{DiPietro2015} and extensions to PDEs of interest to the CFD
community are progressing rapidly.
In the Petrov-Galerkin setting specifically related to this thesis, both the
proposed hybridized discontinuous
Petrov-Galerkin (HDPG)~\cite{Moro2012} and discontinuous Petrov-Galerkin
(DPG)~\cite{Demkowicz2010,Demkowicz2017} methods support the static condensation
property and thus also have this advantage over the standard DG formulations and
the potential to remain competitive with the DG methods for implicit problems
despite requiring more expensive element local operations.