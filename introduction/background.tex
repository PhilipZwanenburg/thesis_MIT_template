\section{Background}
\label{sec:intro_background}

The discontinuous Galerkin (DG) method, initially proposed by Reed et
al.~\cite{Reed1973} and subsequently analyzed for the solution of systems of
conservation
laws~\cite{Cockburn1991,Cockburn1989a,Cockburn1989b,Cockburn1990,Cockburn1998},
has become the most popular choice of high-order scheme in the CFD community.
Despite its widespread usage, there are still several major issues related to
the standard DG scheme:
\begin{itemize}
\item High computational complexity with increasing order of accuracy;
\item Difficulty in generating meshes for complex geometrical objects as well as
  in converting low-order meshes provided by standard mesh generators to
  high-order meshes;
\item In the context of hyperbolic or convection-dominated PDEs, the usage of
  potentially improper stabilization using specific numerical fluxes.
\end{itemize}

Using the DG method as the reference point, we proceed with a discussion of
state-of-the-art advances for tackling the issues listed above with the aim
of highlighting that the DG method may be far from the best choice when
solving systems of hyperbolic conservation laws.

\subsection{Computational Complexity}
