\section{Background}
\label{sec:intro_background}

The discontinuous Galerkin (DG) method, initially proposed by Reed et
al.~\cite{Reed1973} and subsequently analyzed for the solution of systems of
conservation
laws~\cite{Cockburn1991,Cockburn1989a,Cockburn1989b,Cockburn1990,Cockburn1998},
has become the most popular choice of high-order scheme in the CFD community.
Despite its widespread usage, there are still several major issues related to
the standard DG scheme:
\begin{itemize}
\item High computational complexity with increasing order of accuracy;
\item Difficulty in generating meshes for complex geometrical objects as well as
  in converting low-order meshes provided by standard mesh generators to
  high-order meshes;
\item In the context of hyperbolic or convection-dominated PDEs, the usage of
  potentially improper stabilization using specific numerical fluxes.
\end{itemize}

Using the DG method as the reference point, we proceed with a discussion of
state-of-the-art advances for tackling the issues listed above with the aim
of highlighting that the DG method may be far from the best choice when
solving systems of hyperbolic conservation laws.

\subsection{Computational Complexity}
\label{sec:background_comp_comp}

Significant progress has been made with regard to the DG method's computational
complexity, notably through the exploitation of sum factorization techniques,
originally proposed by Orszag~\cite{Orszag1980} and now employed in the
general elements using collapsed tensor-product spaces by Karniadakis et
al.~\cite{Karniadakis1999}. When explicit methods are used, the sum
factorization technique reduces the growth in computational complexity from
$O(N^{2d})$ to $O(N^{d+1})$ where $d$ is the dimension of the problem and $N =
P+1$ where $P$ is the maximal polynomial degree of the basis functions used to
represent the solution; savings for implicit formulations are even greater.

Alternatively, in the case of using implicit methods, the appropriate
decomposition of element bases into corner,
edge, face, and volume modes allows for the possibility of statically condensing out
the volume modes, significantly reducing the growth rate of globally coupled DOF
as the degree of the solution is increased in global matrix inversion stages.
In practice, this choice of polynomial basis representation is rarely used due
to the advantages associated with orthonormal, interpolatory or everywhere
positive basis functions and this reduction in DOF is not possible. This
provided the motivation for the development of what are termed
hybridizable methods where schemes were specifically designed such that global
coupling of the solution occurs only through unknowns defined on element faces,
such that the size of the global system scales in proportion to the number of
face unknowns, exactly corresponding to the static condensation of volume
unknowns.

The hybridizable discontinuous Galerkin (HDG) method was initially
introduced in the context of elliptic equations~\cite{Cockburn2009}, but was
quickly generalized to more complex partial differential equations, notably
to the compressible Navier-Stokes equations~\cite{Peraire2010}.
More recently, a generalized framework called the Hybrid High-Order method which
is able to recover specific HDG formulations with superoptimal convergence for
elliptic problems and which supports general polytopal elements was
proposed~\cite{DiPietro2015} and extensions to PDEs of interest to the CFD
community are progressing rapidly.
In the Petrov-Galerkin setting specifically related to this thesis, both the
proposed hybridized discontinuous
Petrov-Galerkin (HDPG)~\cite{Moro2012} and discontinuous Petrov-Galerkin
(DPG)~\cite{Demkowicz2010,Demkowicz2017} methods support the static condensation
property and thus also have this advantage over the standard DG formulations and
the potential to remain competitive with the DG methods for implicit problems
despite requiring more expensive element local operations.
\subsection{Treatment of Complex Geometry}
\label{sec:background_complex_geom}

The proper treatment of complex geometry in high-order finite element methods has been
shown to be crucial. In the seminal work on the topic in the context of elliptic
PDEs, it was proven that curved geometry had to be represented
isoparametrically, i.e. in a polynomial space having the same degree as the
solution, with specific constraints on the polynomial representation employed
for optimal convergence to be obtainable~\cite[Theorem
\makeblue{5}]{Ciarlet1972}.
While errors due to improper geometry representation may only begin to manifest
themselves at very fine levels of solution resolution, it is precisely the goal
of high-order methods to achieve these levels of accuracy. Further, if these
geometric errors result in decreasing convergence rates of high-order methods,
then the increase in the computational complexity would be incurred with no
additional benefit.

As high-order meshes are commonly generated from the degree elevation of
initially linear meshes, the manner in which the projection of the initially
straight-sided mesh to the curvilinear domain is performed is of critical importance.
In the early years of the application of finite element methods to problems
in domains having curved boundaries, many seemingly disjointed strategies
were proposed to achieve the correct polynomial geometry representation satisfying the
necessary constraints. Assuming suitable placement of geometry interpolation
nodes along the curved boundary, the transfer of the curved face representation
to the volume geometry nodes was achieved using transfinite blending function
interpolation, first proposed by Gordon et al. for tensor-product
elements~\cite{Gordon1973} and subsequently generalized to simplex
elements in both
two~\cite{Nielson1979,Haber1981,Szabo1991,Lacombe1988,Dey1997,Xie2013} and
three~\cite{Lenoir1986} dimensions.
Assuming that a Lagrange polynomial description of the geometry is being used
and that corner nodes are located on the exact curved boundary, the process
proceeds by sequentially projecting straight edge and face nodes to the curved
geometry followed by the application of a blending operation which appropriately
displaces the volume nodes (those not on the domain boundary). Recently, an additional
constraint related to the discrete curvature of the meshed domain was shown to
be necessary for optimal convergence \makered{cite Zwanenburg - Discrete
  Curvature}. In the same article, guidelines were provided concerning the
correct placement of face geometry nodes and a unification of two-dimensional
blending function interpolations and generalization to the three-dimensional
case was presented.

Perhaps surprisingly, numerical results for the Euler equations showed that a
superparametric geometry representation, with polynomial degree one higher than
the solution, was \emph{required} for optimal convergence only when the solution
was represented by a polynomial basis having degree $p = 1$~\cite{Bassi1997};
this numerical result has since been extended to show that the superparametric
geometry representation is in fact required for all polynomial
degrees~\cite{Zwanenburg2017}.
Initially, this phenomenon was explained using the argument that the
that low-order geometry representation results in deterioration of
solution quality as the order of the scheme is increased due to rarefaction
waves being formed at vertices of polygonal mesh
surfaces~\cite{Krivodonova2006}. However, a thorough analysis of the problem was
recently performed in which it was demonstrated that the problem occurs in all
instances in which a boundary condition is used which dependends on a normal
vector computed using the isoparametric geometry representation \makered{cite
  Zwanenburg - Necessity superparametric}. A result of particular interest
arising from this study was the demonstration that the use of exact normal
vectors in combination with isoparametric volume metric terms does not remedy
the problem due to the violation of discrete metric identities resulting in a
high-order conservation error of the same magnitude as that introduced by the
isoparametric normal vectors. This represents the high-order analogue
of the violation of free-stream preservation, which has been shown to be
avoidable by computing metrics according to an elegant curl-formulation by
Kopriva~\cite{Kopriva2006}.

It is important to emphasize that the discussion above assumed that a valid linear
mesh could initially be generated for the complex geometry to be modelled in the
simulation. In fact, it has recently been estimated that approximately 80\% of
overall analysis time in the aerospace industry is devoted to (linear) mesh
generation~\cite{Hughes2005}, resulting in serious challenges when attempting to
interface between simulated results and the
Computer Aided Design (CAD) model, for example. This motivated the formulation
of isogeometric analysis (IGA) where the solution is represented in the same basis as
the CAD geometry (non-polynomial), allowing for perfect geometric representation at any level of
mesh refinement and seamless interfacing with the CAD model~\cite{Hughes2005}.
While the competitiveness of this new approach with existing methods has been
demonstrated in numerous academic benchmark test cases in both structural and fluid
mechanics, its general applicability to test cases having sufficient geometric
complexity to be relevant to the CFD industry is still in question. Further, the
extension of the approximation theory results from the polynomial context discussed
above to the most popular IGA setting employing non-uniform rational spline as
basis functions is still in its infancy.
\subsection{Petrov-Galerkin Approaches for Stabilization}
\label{sec:background_pg_stabilization}

The Galerkin method was originally developed for the solution of PDEs in
structural mechanics in which it can be shown to provide optimal results based
on the implicit relationship betwen the resulting variational formulation and
the minimization of an energy functional.
In the setting of convection-dominated or purely hyperbolic PDEs, the optimal
convergence result is lost and it is now well understood that additional
stabilization is required.
In the linear setting, the degradation in solution accuracy,
generally manifested through the development of unphysical oscillations
throughout the domain, can be formally explained by appealing the approximation
result provided by Babu\v{s}ka's theorem~\cite[Theorem \makeblue{2.2}]{Babuska1971}
which states that the Galerkin error for a well-posed problem satisfies
\begin{align}
||u-u_h||_{\fspace{U}} \le \left( 1 + \frac{M}{\gamma_h} \right) \inf_{w_h \in \fspace{U}_h} ||u-w_h||_{\fspace{U}},
\end{align}

where $M$ and $\gamma_h$ are the continuity and discrete inf-sup constants. The
use of improper stabilization can be directly linked to the vanishing of the
discrete inf-sup constant. This result is not directly generalizable to
the context of the nonlinear PDEs of interest for CFD, but it is still expected
that employing a formulation in which the discrete inf-sup constant is increased
should result in improved stability in a wide range of contexts.

Through the recently popularized interpretation of numerical flux induced
stabilization associated with the DG-type methods through a suitable
penalization of interface jumps~\cite{Brezzi2004,DiPietro2011} it is possible to
derived stability estimates providing explicit representations for the
continuity and inf-sup constants. However, it is not generally possible to
obtain optimal stabilization in this setting (maximizing the discrete inf-sup
constant) nor does the norm used to measure the solution,
$||\cdot||_{\fspace{U}}$, necessarily correspond to that which is most
desirable (notably the $\fspace{L}^2$ norm).

Several methods introducing the
stabilization through the variation of the test space, resulting in a
Petrov-Galerkin framework, have been shown to have the flexibility to meet these
objectives.
The first attempt at the introduction of stabilization through the variation of
the test space was provided in the context of continuous finite element
solutions through the use of residual-based stabilization in the form of the
Streamline Upwind Petrov Galerkin (SUPG) formulation~\cite{Brooks1982}.
It was subsequently shown by the same author that the SUPG formulation was in
essence complementing the solution space with information related to the
fine-scale (unresolvable by the current mesh) Green's function in the context of
the variational multiscale method~\cite{Hughes1998,Hughes2007}.
Approaching the problem from the functional analysis setting discussed above,
the discontinuous Petrov-Galerkin method with optimal test functions was
introduced where the test space is chosen specifically to achieve the supremum
in the discrete inf-sup condition~\cite{Demkowicz2010,Demkowicz2017}; in
essence, the test space is chosen for its good stability properties as opposed
to good approximation properties required for the trial space.
In the CFD setting,
a further emphasized advantage is that the formulation naturally precludes the
need for the formulation of numerical fluxes, requiring the selection of a norm
for the test space instead, which has the potential of being more naturally
selected based on the terms present in the variational form.
A hybridized discontinuous Petrov-Galerkin method~\cite{Moro2012} has also been
proposed as a blend between the HDG and DPG schemes with the goal of retaining
the minimally number of globally coupled DOF while retaining the optimal
stability properties of the DPG scheme. Several subsequent investigations have
focused on attempting to find optimal test norms defined in the sense of the
computed solution being given by the $\fspace{L}^2$ projection of the exact
solution. The investigation was initially undertaken by Bui-Thanh et
al.~\cite{BuiThanh2013} resulting in an impractical method as a result of the
test functions having global support, but eventually reformulated by Brunken et
al.~\cite{Brunken2018} in what will be referred to as the optimal trial
Petrov-Galerkin (OPG) method such that a practical method to achieve this goal was
obtained.

The improved stability of the Petrov-Galerkin methods thus has the potential to reduce the need
for additional ad-hoc approaches where oscillations are popularly suppressed
either through the addition of artificial viscosity or by limiting the solution
using a modal filtering approach. By minimizing modifications to the computed
solution, it is then expected that convergence to the nonlinear solution would
be more robust on underresolved meshes and also that adaptation mechanisms
relying on a posteriori error indicators would have a reduced tendency to refine
the discrete space in incorrect regions as a result of the introduced solution regularization.
The DPG and HDPG methods have both been shown to be able to converge in cases
where DG-type methods have failed, generally in the presence of the
solution having large gradients when no additional stabilization is added.
However, results have generally shown that DPG solutions possess qualitatively
similar oscillations to those present when using the DG-type methods when the
element P\'eclet number is on the order of $\mathcal{O}(10)$, only one order
of magnitude greater than that leading to oscillations for the DG methods.
Further investigation and comparison is thus required in order to establish whether
the added cost is justified.