Following the notation of Pletcher et al.~\cite[Chapter \makeblue{5}]{Pletcher1997}, the continuity, Navier-Stokes and energy equations with source terms neglected are given by
\begin{align} \label{eq:NavierStokes_std}
\frac{\partial \varv{w}}{\partial t} + 
\nabla \cdot \left( \varv{F^i}(\varv{w}) - \varv{F^v}(\varv{w},\varv{Q}) \right) = \vect{0},
\end{align}

where the vector of conservative variables and its gradients are defined as
\begin{alignat}{3}
\varv{w} & \coloneqq \begin{bmatrix} \rho & \rho\varv{v} & E \end{bmatrix}
&& \in \mathbb{R}^{d+2} \nonumber \\
\varv{Q} & \coloneqq \nabla^T \varv{w}
&& \in \mathbb{R}^{d+2} \times \mathbb{R}^{d}, \label{eq:Gradients}
\end{alignat}

where $d$ is the problem dimension, and where the inviscid and viscous fluxes are defined as
\begin{alignat}{4}
&& \varv{F^i}(\varv{w}) & \coloneqq
\begin{bmatrix} \rho\varv{v}^T & \rho \varv{v}^T \varv{v} + p \varv{I} & (E+p) \varv{v}^T \end{bmatrix}
&& \in \mathbb{R}^{d+2} \times \mathbb{R}^{d}, \label{eq:F_inviscid} \\
&& \varv{F^v}(\varv{w},\varv{Q}) & \coloneqq
\begin{bmatrix} \vect{0}^T & \varv{\Pi} & \varv{\Pi} \varv{v}^T - \varv{q}^T \end{bmatrix}
&& \in \mathbb{R}^{d+2} \times \mathbb{R}^{d}. \label{eq:F_viscous}
\end{alignat}

The various symbols represent the density, $\rho$, the velocity, $\varv{v}$, the total energy per unit volume, $E$, the pressure, $p$, the stress tensor, $\varv{\Pi}$ and the energy flux, $\varv{q}$. The pressure is defined according to the equation of state for a calorically ideal gas,
\begin{align*}
p 
=
(\gamma-1) \left( E - \frac{1}{2} \rho \varv{v} \varv{v}^T \right)
\coloneqq
(\gamma-1) \rho e,\ \gamma = \frac{c_p}{c_v},\ c_v = \frac{R_g}{\gamma-1},\ c_p = \frac{\gamma R_g}{\gamma-1},
\end{align*}

where $e$ represents the specific internal energy, $R_g$ is the gas constant and the specific heats at constant volume, $c_v$, and at constant pressure, $c_p$, are constant. The stress tensor is defined as
\begin{align*}
\varv{\Pi} = 2\mu \left( \varv{D} - \frac{1}{3} \nabla \cdot \varv{v} \varv{I} \right),\ \varv{D} \coloneqq \frac{1}{2} \left(\nabla^T \varv{v} + \left(\nabla^T \varv{v}\right)^T \right),
\end{align*}

where $\mu$ is the coefficient of shear viscosity \makered{(Add comment about how $\mu$ is determined (Sutherland, p.259 pletcher(1997)))} and where the coefficient of bulk viscosity was assumed to be zero. Finally, the energy flux is defined by
\begin{align*}
\varv{q} = \kappa \nabla T,
\end{align*}

where $T$ represents the temperature and
\begin{align*}
\kappa = \frac{c_p \mu}{Pr},
\end{align*}

with $Pr$ representing the Prandtl number. In the case of the Euler equations, the contribution of the viscous flux is neglected.
