\chapter{Methodology}

In this section, the governing equations of fluid mechanics and heat transfer, as well as the associated discretizations and boundary conditions employed are outlined. As this work is concerned with the solution of these equations through variants of the finite element method, we also outline the spaces used for the discretization.

Row-vector notation is assumed throughout with the following notation employed:

\begin{tabular}{| lll |}
\hline
Object & Description & Example \\
\hline
Scalar variable & italic & $\vars{a}$ \\
Vector variable & italic boldface lowercase & $\varv{a}$ \\
Second-order tensor variable & italic boldface uppercase & $\varv{A}$ \\
Vector & boldface lowercase & $\vect{a}$ \\
Matrix & boldface uppercase & $\mat{A}$ \\
Spaces & calligraphic uppercase & $\mathcal{A}$ \\
\hline
\end{tabular}


\section{Governing Equations} \label{sec:governing_eqns_NS}
Following the notation of Pletcher et al.~\cite[Chapter \makeblue{5}]{Pletcher1997}, the continuity, Navier-Stokes and energy equations with source terms neglected are given by
\begin{align} \label{eq:NavierStokes_std}
\frac{\partial \varv{w}}{\partial t} + 
\nabla \cdot \left( \varv{F^i}(\varv{w}) - \varv{F^v}(\varv{w},\varv{Q}) \right) = \vect{0},
\end{align}

where the vector of conservative variables and its gradients are defined as
\begin{alignat}{3}
\varv{w} & \coloneqq \begin{bmatrix} \rho & \rho\varv{v} & E \end{bmatrix}
&& \in \mathbb{R}^{d+2} \nonumber \\
\varv{Q} & \coloneqq \nabla^T \varv{w}
&& \in \mathbb{R}^{d+2} \times \mathbb{R}^{d}, \label{eq:Gradients}
\end{alignat}

where $d$ is the problem dimension, and where the inviscid and viscous fluxes are defined as
\begin{alignat}{4}
&& \varv{F^i}(\varv{w}) & \coloneqq
\begin{bmatrix} \rho\varv{v}^T & \rho \varv{v}^T \varv{v} + p \varv{I} & (E+p) \varv{v}^T \end{bmatrix}
&& \in \mathbb{R}^{d+2} \times \mathbb{R}^{d}, \label{eq:F_inviscid} \\
&& \varv{F^v}(\varv{w},\varv{Q}) & \coloneqq
\begin{bmatrix} \vect{0}^T & \varv{\Pi} & \varv{\Pi} \varv{v}^T - \varv{q}^T \end{bmatrix}
&& \in \mathbb{R}^{d+2} \times \mathbb{R}^{d}. \label{eq:F_viscous}
\end{alignat}

The various symbols represent the density, $\rho$, the velocity, $\varv{v}$, the total energy per unit volume, $E$, the pressure, $p$, the stress tensor, $\varv{\Pi}$ and the energy flux, $\varv{q}$. The pressure is defined according to the equation of state for a calorically ideal gas,
\begin{align*}
p 
=
(\gamma-1) \left( E - \frac{1}{2} \rho \varv{v} \varv{v}^T \right)
\coloneqq
(\gamma-1) \rho e,\ \gamma = \frac{c_p}{c_v},\ c_v = \frac{R_g}{\gamma-1},\ c_p = \frac{\gamma R_g}{\gamma-1},
\end{align*}

where $e$ represents the specific internal energy, $R_g$ is the gas constant and the specific heats at constant volume, $c_v$, and at constant pressure, $c_p$, are constant. The stress tensor is defined as
\begin{align*}
\varv{\Pi} = 2\mu \left( \varv{D} - \frac{1}{3} \nabla \cdot \varv{v} \varv{I} \right),\ \varv{D} \coloneqq \frac{1}{2} \left(\nabla^T \varv{v} + \left(\nabla^T \varv{v}\right)^T \right),
\end{align*}

where $\mu$ is the coefficient of shear viscosity \makered{(Add comment about how $\mu$ is determined (Sutherland, p.259 pletcher(1997)))} and where the coefficient of bulk viscosity was assumed to be zero. Finally, the energy flux is defined by
\begin{align*}
\varv{q} = \kappa \nabla T,
\end{align*}

where $T$ represents the temperature and
\begin{align*}
\kappa = \frac{c_p \mu}{Pr},
\end{align*}

with $Pr$ representing the Prandtl number. In the case of the Euler equations, the contribution of the viscous flux is neglected.


\section{Discretizations}

\subsection{Preliminaries}
Let $\Omega$ be a bounded simply connected open subset of $\mathbb{R}^d$ with connected Lipschitz boundary $\partial \Omega$ in $\mathbb{R}^{d-1}$. We let $\Omega_h$ denote the disjoint partion of $\Omega$ into ``elements'', $V$, and denote the element boundaries as $\partial V$. Elements and their boundaries are also referred to as volumes and faces respectively. We also define the following volume inner products,
\begin{alignat*}{5}
(a,b)_D & = \int_D ab;\ && a, b \in L^2(D), \\
(\vect{a},\vect{b})_D & = \int_D \vect{a} \cdot \vect{b};\ && \vect{a}, \vect{b} \in L^2(D)^m, \\
(\vect{A},\vect{B})_D & = \int_D \vect{A} : \vect{B};\ && \vect{A}, \vect{B} \in L^2(D)^{m \times d},
\end{alignat*}

where $D$ is a domain in $\mathbb{R}^d$, and where `$:$' denotes the inner product operator for two second-order tensors. Analogous notation is used for face inner products,
\begin{alignat*}{5}
\left< a,b \right>_D & = \int_D ab;\ && a, b \in L^2(D), \\
\left< \vect{a},\vect{b} \right>_D & = \int_D \vect{a} \cdot \vect{b};\ && \vect{a}, \vect{b} \in L^2(D)^m, \\
\left<\vect{A},\vect{B}\right>_D & = \int_D \vect{A} : \vect{B};\ && \vect{A}, \vect{B} \in L^2(D)^{m \times d},
\end{alignat*}

where $D$ is a domain in $\mathbb{R}^{d-1}$. Denoting the polynomial space of order $p$ on domain $D$ as $\mathcal{P}^p(D)$, and letting $n = d+2$, we define the discontinuous discrete solution and gradient approximation spaces as
\begin{align*}
\mathcal{S}_h^v & = \{ \varv{a} \in L^2(\Omega_h)^{n} : \varv{a} |_V \in \mathcal{P}^p(V)^{n}\ \forall V \in \Omega_h \} \\
\mathcal{G}_h^v & = \{ \varv{A} \in L^2(\Omega_h)^{n \times d} : \varv{A} |_V \in \mathcal{P}^p(V)^{n \times d}\ \forall V \in \Omega_h \}.
\end{align*}

We also define discontinuous test spaces
\begin{align*}
\mathcal{W_t}_h^v & = \{ \vect{a_t} \in L^2(\Omega_h)^{n} : \vect{a_t} |_V \in \mathcal{P}^{p_t}(V)^{n}\ \forall V \in \Omega_h \} \\
\mathcal{Q_t}_h^v & = \{ \vect{A_t} \in L^2(\Omega_h)^{n \times d} : \vect{A_t} |_V \in \mathcal{P}^{p_t}(V)^{n \times d}\ \forall V \in \Omega_h \},
\end{align*}

where $p_t \ge p$. \makered{Will need additional spaces for DPG}.

\subsection{Discretized Equations}
To obtain the discrete formulation, we first define a joint flux $\varv{F}(\varv{w},\varv{Q}) \coloneqq \varv{F^i}(\varv{w}) - \varv{F^v}(\varv{w},\varv{Q})$ then integrate~\eqref{eq:Gradients} and~\eqref{eq:NavierStokes_std} with respect to test functions to obtain
\begin{alignat*}{3}
& \left(\vect{Q_t},\varv{Q} \right)_V = \left(\vect{Q_t},\nabla^T \varv{w}\right)_V, && \forall \vect{Q_t} \in \mathcal{Q_t}_h^v = \mathcal{Q}_h^v \\
& \left(\vect{w_t},\frac{\partial \varv{w}}{\partial t} \right)_V + 
\left(\vect{w_t}, \nabla \cdot \varv{F}(\varv{w},\varv{Q}) \right)_V = \vect{0},\ && \forall \vect{w_t} \in \mathcal{W_t}_h^v = \mathcal{W}_h^v.
\end{alignat*}

Integrating by parts twice in the first equation and once in the second and choosing $p_t = p$, such that the approximation and test spaces are the same, results in the discontinuous Galerkin formulation,
\begin{alignat*}{3}
& \left(\vect{Q_t},\varv{Q} \right)_V
=
\left(\vect{Q_t},\nabla^T \varv{w}\right)_V
+
\left<\vect{Q_t},\vect{n} \cdot \left(\varv{w}^*-\varv{w} \right)\right>_{\partial V},\ && \forall \vect{Q_t} \in \mathcal{Q}_h^v \\
& \left(\vect{w_t},\frac{\partial \varv{w}}{\partial t} \right)_V
- 
\left(\vect{w_t}, \nabla \cdot \varv{F}(\varv{w},\varv{Q}) \right)_V
+
\left<\vect{w_t}, \vect{n} \cdot \varv{F}^* \right>_{\partial V}
= \vect{0},\ && \forall \vect{w_t} \in \mathcal{W}_h^v.
\end{alignat*}

where $\vect{n}$ denotes the outward pointing unit normal vector and where $\varv{w}^*$ and $\varv{F}^*$ represent the numerical solution and flux respectively.


\section{Boundary Conditions}
Boundary conditions are imposed weakly through the specification of a `ghost' state for elements in which $V \cap \partial \Omega \ne \{0\}$. The following boundary conditions are supported:

\resizebox{\textwidth}{!}{
\begin{tabular}{lll}
Boundary Condition & Reference(s) & Comments \\
\hline
Riemann Invariant & \cite[section \makeblue{2.2}]{carlson2011} & eq. (\makeblue{14}) should read $c_b = \frac{\gamma-1}{4} (R^+-R^-)$\\
Slip-Wall & \cite[eq. (\makeblue{10})]{krivodonova2006} & \\
Back Pressure (Outflow) & \cite[section \makeblue{2.4}]{carlson2011} & \\
Total Temperature/Pressure (Inflow) & \cite[section \makeblue{2.7}]{carlson2011} & \\
Supersonic Inflow/Outflow & & imposes exact/extrapolated solution\\
No-slip Temperature       & & imposes $\vect{v}$ and $T$\\
No-slip Adiabatic         & & imposes $\vect{v}$ and $\left( \vect{n} \cdot \vect{F(\vect{W},\vect{Q})} \right)_{E} = 0$\\
\makered{No-slip Flux} & &
\end{tabular}
}


